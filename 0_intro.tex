\chapter{Introduction}\label{sec:intro}

We begin with a natural language description of this paper's main mathematical character, which we will here call an \emph{agent}, which one can imagine as an \emph{evolving planner}. The corresponding mathematical object is found below in \eqref{eqn.homset}.

An agent $A$ is given a \emph{task}, and their job is to deliver an \emph{outcome}. They begin by making a plan: invoke a bunch of $A$'s subordinate agents and hand each of them a task of their own, have the subordinates' resulting outcomes determine a different of bunch of $A$'s subordinate agents to hand different tasks to, and so on, until finally after some finite amount of steps, the process terminates and $A$ obtains an outcome. Agent $A$ can learn from whatever is delivered from its subordinates, so that the next time it is given the same task or a similar one, it can make a different plan. 

This same story can take place at any lower (or higher) level in the same way. In other words, each subordinate agent $B$ may also be executing a plan that invokes $B$'s own subordinates, or instead $A$ itself might be a subordinate who was invoked by the plan of a higher-level agent.

We can move from this natural language description of agents and subagents to a mathematical one, using the framework of polynomial functors (also known as \emph{containers} \cite{abbott2005containers,abbot2003categoriesthesis,ahman2014when}). While polynomial functors are defined as ``coproducts of representables $\smset\to\smset$'', we will think of them as \emph{task families} or task-solution interfaces:
\[
\sum_{T:\text{task}}\yon^{\text{Outcome}[T]}
\]
We intuitively think of tasks like ``purchase an airline ticket to Oakland'', for which an outcome is a certain ticket (or a failure to find one). But one can formalize the notion of task as a type $T$ (e.g.\ $T$=``a prime number $n\geq10^{(10^{10})}$ such that $n+2$ is also prime''), and formalize its outcomes as the terms of that type.%
\footnote{In other words, the set of outcomes might be \emph{intensional} rather than \emph{extensional}: you know what kind of outcome you want, but not how many---if any at all---such outcomes actually exist.} The polynomial $p=\sum_{i:I}\yon^{p[i]}$ is an interface for agents that can be given any task $i$ from a set $I$ and in that case can deliver outcomes from the set $p[i]$.

A morphism $\varphi\colon p\to q$ of polynomials can be represented cleanly in dependent type theory: given another polynomial $q=\sum_{j:J}\yon^{q[j]}$, a natural transformation $\varphi$ between them has type
\[
\poly(p,q)\coloneqq\prod_{i:I}\sum_{j:J}\prod_{e:q[j]}\sum_{d:p[i]}1.
\]
We can understand any such element $\varphi:\poly(p,q)$ as a \emph{task delegation}, which is a two-step process: 
\begin{enumerate}
\item For every task $i$ that $p$ can perform, it \emph{assign} a task $j$ that $q$ can perform.
\item For every outcome $b:q[j]$ that $q$ can deliver, it \emph{returns} an outcome that $p$ can deliver.
\end{enumerate}

In this paper, we consider a more general sort of task delegation by invoking two additional notions: a monoidal structure $\vee$ (which we pronounce as ``or'') and the free monad construction $\free_-$. The monoidal structure allows for the possibility of multiple agents acting simultaneously, and the free monad allows for the possibility of a multi-step process to take place before an outcome is returned. In general, we will see that a task in $\free_p$ is a well-founded tree---or flowchart---of tasks from $p$. 

For example, consider the polynomial $\yon^\nn$; it has only one task, for which an outcome is any natural number. A map $\yon^\nn\to\free_{\yon^\nn\vee\yon^\nn}$ delegates the task to two subagents of the same sort. Such a map might assign to the unique task the two-step process that first asks each subagent for a natural number, then asks whoever had the bigger number (or the first subagent if the numbers were equal) to choose a second number, and finally returns the sum to the original agent. 

In Example~\ref{ex:stream-m}, we will define a map $\yon ^2 \to \free_{{\color{cyan}\yon^2} \vee {\color{magenta}\yon^2} \vee \color{teal}{\yon^2}}$ that delegates the task of selecting a bit to three subagents by asking the first two subagents for a bit. If they agree then return that bit. Otherwise, invoke the third subagent as a tie-breaker. This pattern of task delegation corresponds to the following position of $\free_{{\color{cyan}\yon^2} \vee {\color{magenta}\yon^2} \vee \color{teal}{\yon^2}}$. Note that below we use colors to depict the outcomes of the different subagents. 
\[
\begin{tikzpicture}[trees,
   level distance=1.5cm,sibling distance=1.75cm, 
   edge from parent path={(\tikzparentnode) -- (\tikzchildnode)}]
\Tree [.$\bullet$
    \edge node[auto=left] {$(\azero, \bzero)$};  
    [.$\quad$ ]
    \edge node[auto=right] {$(\azero, \bone)$};
    [.$\bullet$
      \edge node[auto=left] {$\czero$};
      [.$\quad$ ] 
      \edge node[auto=right] {$\cone$};
      [.$\quad$ ] 
    ] 
    \edge node[auto=left] {$(\aone, \bzero)$};
    [.$\bullet$
      \edge node[auto=left] {$\czero$};
      [.$\quad$ ] 
      \edge node[auto=right] {$\cone$};
      [.$\quad$ ] 
    ] 
    \edge node[auto=right] {$(\aone, \bone)$};  
    [.$\quad$ ]
    ]
\end{tikzpicture}
\]
We will see that Huffman coding, an efficient data compression technique, is another example.

We thus obtain a language, which can be seen as an accounting system for agents that can each recursively call subagents to perform tasks, as well as for how agents can \emph{learn} from the resulting outcomes. Mathematically, we structure this as a generalization of \emph{dynamic monoidal category}, in the sense of \cite{shapiro2022dynamic}, of which the gradient descent and backpropagation pattern in deep learning is also an example. That paper discussed various examples of categorical structures (categories, monoidal categories, multicategories, etc.) enriched in $\org$, a variant of which was introduced in \cite[Def 2.19]{spivak2021learnersv1}. 

In this paper, we show that the free monad monad $\free_-\colon\poly\to\poly$ extends to a monad on $\org$, one which is furthermore lax monoidal with respect to $\vee$. The Kleisli $\smcat$-enriched category $\org_\free$ serves as a base of enrichment for a more flexible kind of dynamic categorical structures, ones where the subordinate dynamics can occur at \emph{faster timescales} than the higher-level dynamics does. The above natural language description of an agent $A$, assuming it has task family $p$ and that its subordinates have task families $q_1,\ldots,q_k$, shows up in  \cref{def.orgsharp} as the following formal object:
\begin{equation}\label{eqn.homset}
\org^\sharp_\free(p;q_1,\ldots,q_k)\coloneqq\cofree_{[p,\free_{q_1\vee\cdots\vee q_k}]},
\end{equation}
The monoidal product $\vee$ lets any ``bunch'' of agents be chosen at each time step. The free monad $\free$ brings in the planning aspect, a flow-chart for how subordinate outcomes select new subordinate tasks. The internal hom $[-,-]$ lets tasks pass forward and outcomes pass backwards. And the cofree comonad $\cofree$ lets this whole process evolve through time. 

As a very special case, we give the example of the monoid $\org_\free(\yon,\yon)$, where there is only one subordinate and neither the agent nor the subordinate can vary their task or outcome. In this case, the plan or flow-chart just becomes a number of steps to perform and the evolution is blind, just a stream of plans. Thus an element of this monoid is a natural number sequence $s\colon\nn\to\nn$ and the multiplication $s\star t$ is what appears to be a novel method for combining two such sequences, where $s$ provides the ``time-scale'' for summing in $t$; see \cref{ex.sequence_counter}.

We go two steps further. First we describe a $\smcat$-enriched operad \cite{Leinster:2004a} that we denote $\org^\cofree$, corresponding to the cofree comonad comonad $\cofree_-\colon\poly\to\poly$. And second, we provide a $\smcat$-enriched functor $[-,t]\colon\org_\free\op\to\org^\cofree$ for any polynomial monad $t$, which ``converts patterns to the matter they run on'', in the sense of \cite{libkind2024pattern}; see also \cite{Katsumata2020interaction}. For example, taking $t=\yon$, a hierarchical agent which adds numbers returned by its subordinates  would be sent to a machine that takes streams of numbers (the behaviors of the subordinates)  to the stream of their sum (the behavior of the agent). Taking $t=\lott$ to be the monad corresponding to finite-sample-space random variables, the same is true except that an element of randomness is introduced to the behaviors of the subordinates and agent. All this will be made explicit in the text below.


\section*{Plan of the paper}

In \cref{sec:background} we introduce the $\smcat^\sharp$-enriched category $\org^\sharp$ as well as the adjunctions which define the free monad monad $\free$ and the cofree comonad comonad $\cofree$. 
In \cref{sec:orgm}, we define the $\smcat$-enriched operad $\org_\free$ and in \cref{sec:orgc} we define the $\smcat$-enriched operad $\org^\cofree$. 
Finally, in \cref{sec:matter-pattern} we define a $\smcat$-enriched operad functor $[-, t] \colon \org_\free\op \to \org^\cofree$ for any polynomial monad $t$ and give several applications.

\section*{Acknowledgments}
We appreciate helpful conversations with C.B. Aberl\'e, who suggested our stream-processing example, \cref{ex:stream-m2}.

\thanksAFOSR{FA9550-23-1-0376}.