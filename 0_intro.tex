\chapter{Introduction}\label{sec:intro}

An agent $A$ is given a \emph{task}, and their job is to deliver an \emph{outcome}. They begin by making a plan: invoke a bunch of $A$'s subordinate agents and hand each of them a task of their own, have the subordinates' resulting outcomes determine a different of bunch of $A$'s subordinate agents to hand different tasks to, and so on, until finally after some finite amount of time, the process terminates and $A$ obtains an outcome. Agent $A$ can learn from whatever is delivered from its subordinates, so that the next time it is given the same task or a similar one, it can make a different plan. This same story can take place at any lower (or higher) level in the same way. In other words, each subordinate agent $B$ may also be executing a plan that invokes $B$'s own subordinates, or instead $A$ itself might be a subordinate who was invoked by the plan of a higher-level agent.

We can move from this natural language description of agents and subagents to a mathematical one, using the framework of polynomial functors. While polynomial functors are defined as ``coproducts of representables $\smset\to\smset$'', we will think of them as task-solution interfaces.
\[
\sum_{T:\text{task}}\yon^{\text{Outcome}[T]}
\]
We intuitively think of tasks like ``purchase an airline ticket to Oakland'', for which an outcome is a certain ticket (or a failure to find one). But one can formalize the notion of task as a type $T$ (e.g.\ $T$=``a prime number $n\geq10^{(10^{10})}$ such that $n+2$ is also prime''), and formalize its outcomes as the terms of that type.%
\footnote{In other words, the set of outcomes might be \emph{intensional} rather than \emph{extensional}: you know what kind of outcome you want, but not how many---if any at all---such outcomes actually exist.} The polynomial $p=\sum_{i:I}\yon^{p[i]}$ is an interface for agents that can be given any task $i$ from a set $I$ and in that case can deliver outcomes from the set $p[i]$.

A morphism $\varphi\colon p\to q$ of polynomials can be represented cleanly in dependent type theory: given another polynomial $q=\sum_{j:J}\yon^{q[j]}$, a natural transformation $\varphi$ between them has type
\[
\poly(p,q)\coloneqq\prod_{i:I}\sum_{j:J}\prod_{e:q[j]}\sum_{d:p[i]}1.
\]
We can understand any such element $\varphi:\poly(p,q)$ as a \emph{task delegation}, which is a two-step process: 
\begin{enumerate}
\item for every task $i$ that $p$ can perform, it \emph{assign} a task $j$ that $q$ can perform, and
\item for every outcome $b:q[j]$ that $q$ can deliver, it \emph{returns} an outcome that $p$ can deliver.
\end{enumerate}

In this paper, we consider a more general sort of task delegation by invoking two additional notions: a monoidal structure $\vee$ (which we pronounce as ``or'') and the free monad construction $\free_-$. The monoidal structure allows for the possibility of multiple agents acting simultaneously, and the free monad allows for the possibility of a multi-step process to take place before an outcome is returned. In general, we will see that a task in $\free_p$ is a well-founded tree---or flowchart---of tasks from $p$. 

For example, consider the polynomial $\yon^\nn$; it has only one task, for which an outcome is any natural number. A map $\yon^\nn\to\free_{\yon^\nn\vee\yon^\nn}$ delegates the task to two subagents of the same sort. Such a map might assign to the unique task the two-step process that first asks each subagent for a natural number, then asks whoever had the bigger number (or the first subagent if the numbers were equal) to choose a second number, and finally returns the sum to the original agent. We will see that Huffman coding, an efficient data compression technique, is another example.

We thus obtain a language, which can be seen as an accounting system for agents that can each recursively call subagents to perform tasks, as well as for how agents can \emph{learn} from the resulting outcomes. Mathematically, we structure this as a generalization of \emph{dynamic monoidal category}, in the sense of \cite{shapiro2022dynamic}, of which the gradient descent and backpropagation pattern in deep learning is also an example. That paper discussed various examples of categorical structures (categories, monoidal categories, multicategories, etc.) enriched in $\org$, a variant of which was introduced in \cite[Def 2.19]{spivak2021learnersv1}. 

In this paper, we show that the free monad monad $\free_-\colon\poly\to\poly$ extends to a monad on $\org$, one which is furthermore lax monoidal with respect to $\vee$. The Kleisli \dnote{Replace ``double'' throughout the intro with XXX.} double category $\org_\free$ serves as a base of enrichment for a more flexible kind of dynamic categorical structures, ones where the subordinate dynamics can occur at \emph{faster timescales} than the higher-level dynamics does.

\dnote{Draw a tree-shaped picture here?}

We go only two steps further. First we describe a dynamic operad that we denote $\org^\cofree$, corresponding to the cofree comonad comonad $\cofree_-\colon\poly\to\poly$. And second, we provide a double functor $[-,t]\colon\org_\free\op\to\org^\cofree$ for any polynomial monad $t$, which ``converts patterns to the matter they run on'', in the sense of \cite{libkind2024pattern}. Taking $t=\yon$, a hierarchical agent which adds up pairs of numbers that its subordinates choose would be sent to a machine that takes streams of numbers and adds up consecutive pairs in the corresponding way. Taking $t=\lott$ to be the monad corresponding to finite-sample-space random variables, the same is true except that an element of randomness is introduced at each stage. All this will be made explicit in the text below.


\section*{Plan of the paper}

In \cref{sec:background} we introduce the $\smcat^\sharp$-enriched category $\org^\sharp$ as well as the adjunctions which define the free monad monad $\free$ and the cofree comonad comonad $\cofree$. 
In \cref{sec:orgm}, we define the XXX $\org_\free$ and in \cref{sec:orgc} we define the XXX $\org^\cofree$. 
Finally, in \cref{sec:matter-pattern} we define a XXX functor $[-, t] \colon \org_\free\op \to \org^\cofree$ for any polynomial monad $t$ and give several applications. 

\section*{Acknowledgments}
We appreciate helpful conversations with C.B. Aberl\'e, who suggested our stream-processing example, \cref{**}.

\thanksAFOSR{FA9550-23-1-0376}.