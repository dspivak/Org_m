\chapter{From matter to pattern}\label{sec:matter-pattern}

The main result of this work is the operad functor $(\org_\free^\sharp)\op \to (\org^\sharp)^\cofree$ defined in ~\cref{thm:matter-pattern} which \emph{turns patterns of delegation into matter (i.e. behaviors)}. In this Section we will prove ~\cref{thm:matter-pattern} and then make sense of this interpretation through examples.

% Following ~\cite{libkind2024pattern} the positions of $\free_p$ are $p$-shaped decision trees which we interpret as patterns of shape $p$. The nodes of such a tree are positions of $p$ (which we interpret as $p$-questions) and branches are directions of $p$ (which we interpret as $p$-answers). Given such a decision tree, we want a way to \textit{run} it. In other words, we want a device that gives the necessary information to 

Throughout let $t$ be a $\tri$-monoid.  There is a unique polynomial map
\begin{equation}\label{eq:p1-vee-p2}
    (p_1 \vee p_2) \otimes \cofree_{[p_1, t]} \otimes \cofree_{[p_2,t]} \to t
\end{equation}
whose components consist of 
\[
    p_1 \otimes p_2 \otimes \cofree_{[p_1, t]} \otimes \cofree_{[p_2,t]} \to p_1 \otimes p_2 \otimes [p_1, t] \otimes [p_2,t] \to t \otimes t  \to t 
\]  and for $i = 1, 2$
\[
    p_i \otimes \cofree_{[p_1, t]} \otimes \cofree_{[p_2,t]} \to p_i \otimes \cofree_{[p_i, t]} \to p_i \otimes [p_i, t] \to t.
\]
Via the $\free$ adjunction, the map in \cref{eq:p1-vee-p2} induces  a polynomial map 
\begin{equation}\label{eq:free-p1-vee-p2}
    \free_{p_1 \vee p_2} \otimes \cofree_{[p_1, t]} \otimes \cofree_{[p_2,t]} \to t .
\end{equation}
For any polynomial $q$, \cref{eq:free-p1-vee-p2} in turn induces a cofunctor
\begin{equation}\label{eq:cofree_q_free_p1_vee_p2}
    \cofree_{[q, \free_{p_1 \vee p_2}]} \otimes \cofree_{[p_1, t]} \otimes \cofree_{[p_2,t]} \to \cofree_{[q,t]}
\end{equation}

Importantly, the polynomial map in \cref{eq:free-p1-vee-p2} satisfies an associativity property which we summarize in \cref{lem:free_p1_vee_p2_associative}.

\begin{lemma}\label{lem:free_p1_vee_p2_associative}
    For polynomials $p_1$, $p_2$, $q_1$, and $q_2$ and for $\tri$-monoids $s$ and $t$, the following diagram commutes.

    % https://q.uiver.app/#q=WzAsNCxbMCwwLCJcXGZyZWVfe3FfMSBcXHZlZSBxXzJ9IFxcb3RpbWVzIFxcY29mcmVlX3tbcV8xLCBzXX1cXG90aW1lcyBcXGNvZnJlZV97W3FfMiwgXFxmcmVlX3twXzEgXFx2ZWUgcF8yfV19IFxcb3RpbWVzIFxcY29mcmVlX3tbcF8xLCB0XzFdfSBcXG90aW1lcyBcXGNvZnJlZV97W3BfMiwgdF8yXX0iXSxbMSwwLCJcXGZyZWVfe3FfMSBcXHZlZSBxXzJ9IFxcb3RpbWVzIFxcY29mcmVlX3tbcV8xLCBzXX0gXFxvdGltZXMgXFxjb2ZyZWVfe1txXzIsIHRfMSBcXG90aW1lcyB0XzJdfSJdLFsxLDEsInMgXFxvdGltZXMgdF8xIFxcb3RpbWVzIHRfMiJdLFswLDEsInMgXFxvdGltZXMgXFxmcmVlX3twXzEgXFx2ZWUgcF8yfSBcXG90aW1lcyBcXGNvZnJlZV97W3BfMSwgdF8xXX0gXFxvdGltZXMgXFxjb2ZyZWVfe1twXzIsIHRfMl19Il0sWzAsMV0sWzEsMl0sWzMsMl0sWzAsM11d
\[\begin{tikzcd}
	{\free_{q_1 \vee q_2} \otimes \cofree_{[q_1, s]}\otimes \cofree_{[q_2, \free_{p_1 \vee p_2}]} \otimes \cofree_{[p_1, t]} \otimes \cofree_{[p_2, t]}} & {\free_{q_1 \vee q_2} \otimes \cofree_{[q_1, s]} \otimes \cofree_{[q_2, t \otimes t]}} \\
	{s \otimes \free_{p_1 \vee p_2} \otimes \cofree_{[p_1, t]} \otimes \cofree_{[p_2, t]}} & {s \otimes t \otimes t}
	\arrow[from=1-1, to=1-2]
	\arrow[from=1-1, to=2-1]
	\arrow[from=1-2, to=2-2]
	\arrow[from=2-1, to=2-2]
\end{tikzcd}\]
    The clockwise map is an application of \cref{eq:cofree_q_free_p1_vee_p2} followed by an application of \cref{eq:free-p1-vee-p2}. The counter-clockwise map is defined by two applications of \cref{eq:free-p1-vee-p2}. 
\end{lemma}
\begin{proof}
    It suffices to show that the diagram commutes when precomposed with the inclusions $q_i \to \free_{q_1 \vee q_2}$ and $q_1 \otimes q_2 \to \free_{q_1 \vee q_2}$.

    For the inclusion $q_1 \to \free_{q_1 \vee q_2}$ it suffices to show that the following diagram commutes.
    % https://q.uiver.app/#q=WzAsOCxbMCwwLCJxXzEgXFxvdGltZXMgXFxjb2ZyZWVfe1txXzEsIHNdfVxcb3RpbWVzIFxcY29mcmVlX3tbcV8yLCBcXGZyZWVfe3BfMSBcXHZlZSBwXzJ9XX0gXFxvdGltZXMgXFxjb2ZyZWVfe1twXzEsIHRdfSBcXG90aW1lcyBcXGNvZnJlZV97W3BfMiwgdF19Il0sWzEsMCwicV8xIFxcb3RpbWVzIFxcY29mcmVlX3tbcV8xLCBzXX0gXFxvdGltZXMgXFxjb2ZyZWVfe1txXzIsIHRfMSBcXG90aW1lcyB0XzJdfSJdLFsxLDEsInFfMSBcXG90aW1lcyBbcV8xLCBzXSBcXG90aW1lcyBcXGNvZnJlZV97W3FfMiwgdF19Il0sWzAsMSwicV8xIFxcb3RpbWVzIFtxXzEsIHNdIFxcb3RpbWVzIFxcY29mcmVlX3tbcV8xLCBcXGZyZWVfe3BfMSBcXHZlZSBwXzJ9XX0gXFxvdGltZXMgXFxjb2ZyZWVfe1twXzEsIHRdfSBcXG90aW1lcyBcXGNvZnJlZV97W3BfMiwgdF19Il0sWzAsMiwicyBcXG90aW1lcyBcXGNvZnJlZV97W3BfMSwgdF19IFxcb3RpbWVzIFxcY29mcmVlX3tbcF8yLCB0XX0iXSxbMSwyLCJzIl0sWzAsMywicyBcXG90aW1lcyBcXGZyZWVfe3BfMSBcXHZlZSBwXzJ9IFxcb3RpbWVzIFxcY29mcmVlX3tbcF8xLCB0XX0gXFxvdGltZXMgXFxjb2ZyZWVfe1twXzIsIHRdfSJdLFsxLDMsInMgXFxvdGltZXMgdCJdLFswLDFdLFsxLDJdLFszLDJdLFswLDNdLFszLDRdLFsyLDVdLFs0LDVdLFs2LDddLFs0LDZdLFs1LDddXQ==
\[\begin{tikzcd}
	{q_1 \otimes \cofree_{[q_1, s]}\otimes \cofree_{[q_2, \free_{p_1 \vee p_2}]} \otimes \cofree_{[p_1, t]} \otimes \cofree_{[p_2, t]}} & {q_1 \otimes \cofree_{[q_1, s]} \otimes \cofree_{[q_2, t_1 \otimes t_2]}} \\
	{q_1 \otimes [q_1, s] \otimes \cofree_{[q_1, \free_{p_1 \vee p_2}]} \otimes \cofree_{[p_1, t]} \otimes \cofree_{[p_2, t]}} & {q_1 \otimes [q_1, s] \otimes \cofree_{[q_2, t]}} \\
	{s \otimes \cofree_{[p_1, t]} \otimes \cofree_{[p_2, t]}} & s \\
	{s \otimes \free_{p_1 \vee p_2} \otimes \cofree_{[p_1, t]} \otimes \cofree_{[p_2, t]}} & {s \otimes t}
	\arrow[from=1-1, to=1-2]
	\arrow[from=1-1, to=2-1]
	\arrow[from=1-2, to=2-2]
	\arrow[from=2-1, to=2-2]
	\arrow[from=2-1, to=3-1]
	\arrow[from=2-2, to=3-2]
	\arrow[from=3-1, to=3-2]
	\arrow[from=3-1, to=4-1]
	\arrow[from=3-2, to=4-2]
	\arrow[from=4-1, to=4-2]
\end{tikzcd}\]
It is obvious that the top square commutes. The middle square commutes because the map in \cref{eq:cofree_q_free_p1_vee_p2} is a map of cofunctors and hence preserves the counit. The bottom square commutes because the definition of \cref{eq:free-p1-vee-p2} implies that  $\free_{p_1 \vee p_2} \to [ \cofree_{[p_1, t]} \otimes \cofree_{[p_2, t]}, t ]$ is a map of $\tri$-monoids and hence preserves the unit.

For the inclusion $q_2 \to \free_{q_1 \vee q_2}$ and $q_1 \otimes q_2 \to \free_{q_1 \vee q_2}$, it suffices to show that the following diagram commutes. 
% https://q.uiver.app/#q=WzAsNixbMCwwLCJxXzIgXFxvdGltZXMgXFxjb2ZyZWVfe1txXzIsIFxcZnJlZV97cF8xIFxcdmVlIHBfMn1dfSBcXG90aW1lcyBcXGNvZnJlZV97W3BfMSwgdF19IFxcb3RpbWVzIFxcY29mcmVlX3tbcF8yLCB0XX0iXSxbMSwwLCJxXzIgXFxvdGltZXMgXFxjb2ZyZWVfe1txXzIsIHRdfSJdLFswLDEsInFfMiBcXG90aW1lcyBbcV8yLCBcXGZyZWVfe3BfMSBcXHZlZSBwXzJ9XSBcXG90aW1lcyBcXGNvZnJlZV97W3BfMSwgdF19IFxcb3RpbWVzIFxcY29mcmVlX3tbcF8yLCB0XX0iXSxbMCwyLCJcXGZyZWVfe3BfMSBcXHZlZSBwXzJ9XFxvdGltZXMgXFxjb2ZyZWVfe1twXzEsIHRdfSBcXG90aW1lcyBcXGNvZnJlZV97W3BfMiwgdF19Il0sWzEsMSwicV8yIFxcb3RpbWVzIFtxXzIsIHRdIl0sWzEsMiwidCJdLFswLDJdLFsyLDNdLFszLDVdLFs0LDVdLFsxLDRdLFswLDFdXQ==
\[\begin{tikzcd}
	{q_2 \otimes \cofree_{[q_2, \free_{p_1 \vee p_2}]} \otimes \cofree_{[p_1, t]} \otimes \cofree_{[p_2, t]}} & {q_2 \otimes \cofree_{[q_2, t]}} \\
	{q_2 \otimes [q_2, \free_{p_1 \vee p_2}] \otimes \cofree_{[p_1, t]} \otimes \cofree_{[p_2, t]}} & {q_2 \otimes [q_2, t]} \\
	{\free_{p_1 \vee p_2}\otimes \cofree_{[p_1, t]} \otimes \cofree_{[p_2, t]}} & t
	\arrow[from=1-1, to=1-2]
	\arrow[from=1-1, to=2-1]
	\arrow[from=1-2, to=2-2]
	\arrow[from=2-1, to=3-1]
	\arrow[from=2-2, to=3-2]
	\arrow[from=3-1, to=3-2]
\end{tikzcd}\]
It is immediate from the definition of the map in \cref{eq:cofree_q_free_p1_vee_p2}.

\end{proof}

The map in \cref{eq:free-p1-vee-p2} is the key ingredient to defining maps
\begin{equation}\label{eq:free-p1-vee-pn}
    \free_{p_1 \vee \cdots \vee p_n} \otimes \cofree_{[p_1, t]} \otimes \cdots\otimes \cofree_{[p_n, t]} \to t
\end{equation}
by induction.
The base case  $\free_\yon \to t$ is the counit $\mcoun_\yon$ followed by the unit $\yon \to t$ of $t$. For the induction step, the map in \cref{eq:free-p1-vee-pn} induces a retrofunctor \snote{Why is this a retrofunctor?}
\[
    \cofree_{[p_1, t]} \otimes \cdots \otimes \cofree_{[p_n, t]} \to \cofree_{[p_1 \vee \cdots \vee p_n, t]}
\] via the unit $\mun_{p_1 \vee \cdots \vee p_n}\colon p_1 \vee \cdots \vee p_n \to \free_{p_1 \vee \cdots \vee p_n}$.
Composing this map with the map in \cref{eq:free-p1-vee-p2} concludes the induction argument:
\[
    \free_{p_1 \vee \cdots \vee p_n \vee p_{n+1}} \otimes \cofree_{[p_1, t]} \otimes \cdots\otimes \cofree_{[p_n, t]} \otimes \cofree_{[p_{n+1} , t]} 
    \to  \free_{p_1 \vee \cdots \vee p_n \vee p_{n+1}} \otimes \cofree_{[p_1 \vee \cdots \vee p_n, t]} \otimes \cofree_{[p_{n+1} , t]} 
       \to  t .
\] 




\begin{theorem}\label{thm:matter-pattern}
    For any $\tri$-monoid $t$, there is an operad functor 
    \[
    [-,t]\colon (\org_\free^\sharp)\op \to (\org^\sharp)^\cofree
    \]
    which on objects maps a polynomial $p$ to the polynomial $[p,t]$. On morphisms, it is the $\smcat^\sharp$ map
    \[
        \cofree_{[q, \free_{p_1 \vee \cdots \vee p_n}]} \to \cofree_{[\cofree_{[p_1,t]} \otimes \cdots \otimes \cofree_{[p_n, t]}, \cofree_{[q,t]}]}
    \] that is the image under $\cofree \colon \poly \to \smcat^\#$ of the  polynomial map \[ [q, \free_{p_1 \vee \cdots \vee p_n}] \to [\cofree_{[p_1, t]} \otimes \cdots \otimes \cofree_{[p_n, t]} , [q, t]]\] induced by 
    \[
        q \otimes [q, \free_{p_1 \vee \cdots \vee p_n}] \otimes \cofree_{[p_1, t]} \otimes \cdots \otimes \cofree_{[p_n, t]}  \xrightarrow{\eval} \free_{p_1 \vee \cdots \vee p_n} \otimes \cofree_{[p_1, t]} \otimes \cdots \otimes \cofree_{[p_n, t]}  \to t.
    \]
   
    
\end{theorem}
\begin{proof}
    This functor preserves the identity because for all polynomials $p$, the definition of $\modstruct$ along with the zig-zag law of the $\free$ adjunction implies that the following diagram commutes. 

    % https://q.uiver.app/#q=WzAsNixbMCwwLCJxIFxcb3RpbWVzIFxcY29mcmVlX3tbcSwgdF19Il0sWzAsMSwicSBcXG90aW1lcyBbcSwgdF0iXSxbMSwwLCJcXGZyZWVfcSBcXG90aW1lcyBcXGNvZnJlZV97W3EsdF19Il0sWzIsMCwiXFxmcmVlX3txIFxcb3RpbWVzIFtxLHRdfSJdLFszLDAsIlxcZnJlZV90Il0sWzMsMSwidCJdLFswLDEsInEgXFxvdGltZXMgXFxjY291bl97W3EsIHRdfSIsMl0sWzAsMiwiXFxtdW5fcSBcXG90aW1lcyBcXGNvZnJlZV97W3EsdF19Il0sWzEsNSwiXFxldmFsIiwyXSxbNCw1LCJcXG1jb3VuX3QiXSxbMiwzLCJcXG1vZHN0cnVjdF97cSwgW3EsdF19Il0sWzMsNCwiXFxmcmVlX1xcZXZhbCJdXQ==
\[\begin{tikzcd}[column sep=huge]
	{q \otimes \cofree_{[q, t]}} & {\free_q \otimes \cofree_{[q,t]}} & {\free_{q \otimes [q,t]}} & {\free_t} \\
	{q \otimes [q, t]} &&& t
	\arrow["{\mun_q \otimes \cofree_{[q,t]}}", from=1-1, to=1-2]
	\arrow["{q \otimes \ccoun_{[q, t]}}"', from=1-1, to=2-1]
	\arrow["{\modstruct_{q, [q,t]}}", from=1-2, to=1-3]
	\arrow["{\free_\eval}", from=1-3, to=1-4]
	\arrow["{\mcoun_t}", from=1-4, to=2-4]
	\arrow["\eval"', from=2-1, to=2-4]
\end{tikzcd}\]

    To show that this functor preserves composition, by symmetry, it suffices to show that for $n \geq 1$ the following diagram of retrofunctors commutes.

    % https://q.uiver.app/#q=WzAsNCxbMCwwLCJcXGNvZnJlZV97XFxsZWZ0W3IgLCBcXGZyZWVfe1xcYmlndmVlX3tqID0gMX1ebiBxX2p9XFxyaWdodF19IFxcb3RpbWVzIFxcY29mcmVlX3tcXGxlZnRbcV9uICwgXFxmcmVlX3tcXGJpZ3ZlZV97aSA9IDF9Xm0gcF9pfVxccmlnaHRdfSJdLFsxLDAsIlxcY29mcmVlX3tcXGxlZnRbciAsIFxcZnJlZV97XFxsZWZ0KFxcYmlndmVlX3tqID0gMX1ee24tMX0gcV9qXFxyaWdodCkgXFx2ZWUgXFxsZWZ0KFxcYmlndmVlX3tpID0gMX1ebSBwX2lcXHJpZ2h0KX1cXHJpZ2h0XX0iXSxbMCwxLCJcXGNvZnJlZV97XFxsZWZ0W1xcYmlnb3RpbWVzX3tqID0gMX1ebiBcXGNvZnJlZV97W3FfaiwgdF19LCBbciwgdF1cXHJpZ2h0XX0gXFxvdGltZXMgXFxjb2ZyZWVfe1xcbGVmdFtcXGJpZ290aW1lc197aSA9IDF9Xm0gXFxjb2ZyZWVfe1twX2ksIHRdfSwgW3FfbiwgdF1cXHJpZ2h0XX0iXSxbMSwxLCJcXGNvZnJlZV97XFxsZWZ0W1xcbGVmdChcXGJpZ290aW1lc197aiA9IDF9XntuLTF9IFtxX2osdF1cXHJpZ2h0KSBcXG90aW1lcyBcXGxlZnQoXFxiaWdvdGltZXMgX3tpID0gMX1ebSBbcF9pLCB0XVxccmlnaHQpLCBbciAsdF1cXHJpZ2h0XX0iXSxbMCwxLCIiLDEseyJzdHlsZSI6eyJib2R5Ijp7Im5hbWUiOiJiYXJyZWQifX19XSxbMiwzLCIiLDEseyJzdHlsZSI6eyJib2R5Ijp7Im5hbWUiOiJiYXJyZWQifX19XSxbMSwzLCIiLDEseyJzdHlsZSI6eyJib2R5Ijp7Im5hbWUiOiJiYXJyZWQifX19XSxbMCwyLCIiLDEseyJzdHlsZSI6eyJib2R5Ijp7Im5hbWUiOiJiYXJyZWQifX19XV0=
\[\begin{tikzcd}
	{\cofree_{\left[r , \free_{\bigvee_{j = 1}^n q_j}\right]} \otimes \cofree_{\left[q_n , \free_{\bigvee_{i = 1}^m p_i}\right]}} & {\cofree_{\left[r , \free_{\left(\bigvee_{j = 1}^{n-1} q_j\right) \vee \left(\bigvee_{i = 1}^m p_i\right)}\right]}} \\
	{\cofree_{\left[\bigotimes_{j = 1}^n \cofree_{[q_j, t]}, [r, t]\right]} \otimes \cofree_{\left[\bigotimes_{i = 1}^m \cofree_{[p_i, t]}, [q_n, t]\right]}} & {\cofree_{\left[\left(\bigotimes_{j = 1}^{n-1} [q_j,t]\right) \otimes \left(\bigotimes _{i = 1}^m [p_i, t]\right), [r ,t]\right]}}
	\arrow["\shortmid"{marking}, from=1-1, to=1-2]
	\arrow["\shortmid"{marking}, from=1-1, to=2-1]
	\arrow["\shortmid"{marking}, from=1-2, to=2-2]
	\arrow["\shortmid"{marking}, from=2-1, to=2-2]
\end{tikzcd}\]

    For $m = 0$, it is immediate. For $m \geq 1$ consider the diagram on the following page. It is immediate that the top square commutes. The middle square commutes by definition of the upper horizontal map. The bottom square commutes by \cref{lem:free_p1_vee_p2_associative}.
    \newpage
    \begin{sideways}
    % https://q.uiver.app/#q=WzAsOCxbMCwwLCJcXGZyZWVfe1xcYmlndmVlX3tqID0gMX1ebiBxX2p9IFxcb3RpbWVzIFxcbGVmdCggXFxiaWdvdGltZXNfe2ogPSAxfV57biAtIDF9IFxcY29mcmVlX3tbcV9qLCB0XX1cXHJpZ2h0KSBcXG90aW1lcyBcXGNvZnJlZV97W3FfbiwgXFxmcmVlX3tcXGJpZ3ZlZV97aSA9IDF9Xm0gcF9pfV19IFxcb3RpbWVzIFxcbGVmdChcXGJpZ290aW1lc197aSA9IDF9Xm0gXFxjb2ZyZWVfe1twX2ksIHRdfVxccmlnaHQpIl0sWzEsMCwiIHQgXFxvdGltZXMgIFxcZnJlZV97XFxiaWd2ZWVfe2kgPSAxfV5tIHBfaX0gXFxvdGltZXMgXFxsZWZ0KFxcYmlnb3RpbWVzX3tpID0gMX1ebSBcXGNvZnJlZV97W3BfaSwgdF19XFxyaWdodCkiXSxbMCwxLCJcXGZyZWVfe1xcYmlndmVlX3tqID0gMX1ebiBxX2p9IFxcb3RpbWVzIFxcbGVmdCggXFxiaWdvdGltZXNfe2ogPSAxfV57biAtIDF9IFxcY29mcmVlX3tbcV9qLCB0XX1cXHJpZ2h0KSBcXG90aW1lcyBcXGNvZnJlZV97W3FfbiwgXFxmcmVlX3tcXGJpZ3ZlZV97aSA9IDF9Xm0gcF9pfV19IFxcb3RpbWVzIFxcY29mcmVlX3tcXGxlZnRbXFxiaWd2ZWVfe2kgPSAxfV57bS0xfSBwX2ksICB0XFxyaWdodF19IFxcb3RpbWVzIFxcY29mcmVlX3tbcF9tLCB0XX0iXSxbMSwxLCJ0IFxcb3RpbWVzICBcXGZyZWVfe1xcYmlndmVlX3tpID0gMX1ebSBwX2l9IFxcb3RpbWVzIFxcY29mcmVlX3tcXGxlZnRbXFxiaWd2ZWVfe2kgPSAxfV57bS0xfSBwX2ksIHQgXFxyaWdodF19IFxcb3RpbWVzIFxcY29mcmVlX3tbcF9tLCB0XX0iXSxbMCwyLCJcXGZyZWVfe1xcYmlndmVlX3tqID0gMX1ebiBxX2p9IFxcb3RpbWVzIFxcY29mcmVlX3tcXGxlZnRbXFxiaWd2ZWVfe2ogPSAxfV57biAtMX0gcV9qLCB0XFxyaWdodF19IFxcb3RpbWVzIFxcY29mcmVlX3tbcV9uLCBcXGZyZWVfe1xcYmlndmVlX3tpID0gMX1ebSBwX2l9XX0gXFxvdGltZXMgXFxjb2ZyZWVfe1xcbGVmdFtcXGJpZ3ZlZV97aSA9IDF9XnttLTF9IHBfaSwgdCBcXHJpZ2h0XX0gXFxvdGltZXMgXFxjb2ZyZWVfe1twX20sIHRdfSJdLFsxLDIsIiB0IFxcb3RpbWVzICBcXGZyZWVfe1xcYmlndmVlX3tpID0gMX1ebSBwX2l9IFxcb3RpbWVzIFxcY29mcmVlX3tcXGxlZnRbXFxiaWd2ZWVfe2kgPSAxfV57bS0xfSBwX2ksIHRcXHJpZ2h0XX0gXFxvdGltZXMgXFxjb2ZyZWVfe1twX20sIHRdfSJdLFswLDMsInQgXFxvdGltZXMgIFxcZnJlZV97XFxiaWd2ZWVfe2kgPSAxfV5tIHBfaX0gXFxvdGltZXMgXFxjb2ZyZWVfe1xcbGVmdFtcXGJpZ3ZlZV97aSA9IDF9XnttLTF9IHBfaSwgdFxccmlnaHRdfSBcXG90aW1lcyBcXGNvZnJlZV97W3BfbSwgdF19Il0sWzEsMywidCBcXG90aW1lcyB0Il0sWzAsMV0sWzAsMl0sWzEsM10sWzIsM10sWzAsMl0sWzMsNSwiIiwxLHsibGV2ZWwiOjIsInN0eWxlIjp7ImhlYWQiOnsibmFtZSI6Im5vbmUifX19XSxbMiw0XSxbNCw1XSxbNSw3XSxbNiw3XSxbNCw2XV0=
    \begin{tikzcd}[row sep=large]
    	{\free_{\bigvee_{j = 1}^n q_j} \otimes \left( \bigotimes_{j = 1}^{n - 1} \cofree_{[q_j, t]}\right) \otimes \cofree_{[q_n, \free_{\bigvee_{i = 1}^m p_i}]} \otimes \left(\bigotimes_{i = 1}^m \cofree_{[p_i, t]}\right)} & { t \otimes  \free_{\bigvee_{i = 1}^m p_i} \otimes \left(\bigotimes_{i = 1}^m \cofree_{[p_i, t]}\right)} \\
    	{\free_{\bigvee_{j = 1}^n q_j} \otimes \left( \bigotimes_{j = 1}^{n - 1} \cofree_{[q_j, t]}\right) \otimes \cofree_{[q_n, \free_{\bigvee_{i = 1}^m p_i}]} \otimes \cofree_{\left[\bigvee_{i = 1}^{m-1} p_i,  t\right]} \otimes \cofree_{[p_m, t]}} & {t \otimes  \free_{\bigvee_{i = 1}^m p_i} \otimes \cofree_{\left[\bigvee_{i = 1}^{m-1} p_i, t \right]} \otimes \cofree_{[p_m, t]}} \\
    	{\free_{\bigvee_{j = 1}^n q_j} \otimes \cofree_{\left[\bigvee_{j = 1}^{n -1} q_j, t\right]} \otimes \cofree_{[q_n, \free_{\bigvee_{i = 1}^m p_i}]} \otimes \cofree_{\left[\bigvee_{i = 1}^{m-1} p_i, t \right]} \otimes \cofree_{[p_m, t]}} & { t \otimes  \free_{\bigvee_{i = 1}^m p_i} \otimes \cofree_{\left[\bigvee_{i = 1}^{m-1} p_i, t\right]} \otimes \cofree_{[p_m, t]}} \\
    	{t \otimes  \free_{\bigvee_{i = 1}^m p_i} \otimes \cofree_{\left[\bigvee_{i = 1}^{m-1} p_i, t\right]} \otimes \cofree_{[p_m, t]}} & {t \otimes t}
    	\arrow[from=1-1, to=1-2]
    	\arrow[from=1-1, to=2-1]
    	\arrow[from=1-1, to=2-1]
    	\arrow[from=1-2, to=2-2]
    	\arrow[from=2-1, to=2-2]
    	\arrow[from=2-1, to=3-1]
    	\arrow[Rightarrow, no head, from=2-2, to=3-2]
    	\arrow[from=3-1, to=3-2]
    	\arrow[from=3-1, to=4-1]
    	\arrow[from=3-2, to=4-2]
    	\arrow[from=4-1, to=4-2]
    \end{tikzcd}
\end{sideways}

    
\end{proof}


Under the map of enriching categories $\smset^- \colon \smcat^\sharp \to \smcat$, 
\cref{thm:matter-pattern} defines an $\smcat$-enriched operad functor $\org_\free\op \to \org^\cofree$ which turns a 
$[q, \free_{p_1 \vee \cdots \vee p_m}]$-coalgebra into a 
$[\cofree_{[p_1, t]} \otimes \cdots \otimes \cofree_{[p_m, t]}, [q, t]]$-coalgebra. 

\begin{example}\label{ex:stream-m2}
    Consider the $[\yon^2, \free_{\alice \vee \bob \vee \carmen}]$-coalgebras defined in \cref{ex:stream-m}. Under the $\smcat$-enriched operad functor $[-,t]\colon\org_\free\op \to \org^\cofree$, each one defines a $[\cofree_{[\alice, t] \otimes \cofree_{[\bob, t]} \otimes \cofree_{[\carmen, t]}}, [\yon^2, t])$-coalgebra which is equivalent to a polynomial map 
    \begin{equation}\label{eq:matter}
        S\yon^S \otimes \yon^2 \otimes \cofree_{[\alice, t]} \otimes \cofree_{[\bob, t]} \otimes \cofree_{[\carmen, t]} \to t.
    \end{equation}
        
    In general $[\yon^A , \yon] \iso A\yon$. So for $t = \yon$, the positions of the polynomial $\cofree_{[\alice, t]} = \cofree_{2\yon}$ are streams of $0$s and $1$s. A particular choice of stream $\yon \to \cofree_{[\alice, t]}$ represents Alice's behavior, in other words which bits she will respond with when invoked. Likewise for $\cofree_{[\bob, t]}$  and $\cofree_{[\carmen, t]}$.  Given behaviors for Alice, Bob, and Carmen and a starting state in $S$, \cref{eq:matter} defines a stream $\yon \to \cofree_{[\yon^2, t]}$. A token of this stream is produced by applying the results of the Alice, Bob, and Carmen streams to the decision tree in $\free_{\alice \vee \bob \vee \carmen}$ defined by the state. For each token, zero or one (or in the case of more complicated decision trees, multiple) tokens of Alice, Bob, and Carmen's streams are consumed. \cref{fig:abc-stream} gives an example of the map from behaviors of Alice, Bob, and Carmen for the delegation pattern described in \cref{fig:carmen-tie-breaks}.

    \begin{figure}
        \begin{subfigure}[b]{0.32\textwidth}
             
\begin{tabular}{r|l}
     Alice & $0|000 \cdots $ \\
     Bob & $0|101 \cdots $ \\
     Carmen & $|1011 \cdots$ \\ \hline
     Result & $0$\\
\end{tabular}
             \caption{}
         \end{subfigure}
         \begin{subfigure}[b]{0.32\textwidth}
             \begin{tabular}{r|l}
     Alice & $00|00 \cdots $ \\
     Bob & $01|01 \cdots $ \\
     Carmen & $1|011 \cdots$ \\\hline
     Result & $01$\\
\end{tabular}
             \caption{}
         \end{subfigure}
         \begin{subfigure}[b]{0.32\textwidth}
             \begin{tabular}{r|l}
     Alice & $000|0 \cdots $ \\
     Bob & $010|1 \cdots $ \\
     Carmen & $1|011 \cdots$ \\ \hline
     Result & $010$\\
\end{tabular}
             \caption{}
         \end{subfigure}
        \caption{The functor $[-, \yon ] \colon \org_\free\op \to \org^\cofree$ transforms the delegation pattern in \cref{fig:carmen-tie-breaks} in to a way of determining outcomes from behaviors of Alice, Bob, and Carmen. For example, given the shown streams of behavior, this delegation pattern  first produces a $0$ (see (a)) because Alice and Bob agreed. This outcome consumes the first token in Alice's stream and Bob's stream. It consumes none of the tokens in Carmen's stream, because Carmen was not invoked to tie-break. Next, it produces a $1$ (see (b)) by Carmen's tie-break. These first two outcomes consume  the first two tokens in Alice's stream and Bob's stream. It consumes the first token in Carmen's stream. Next, it produces a $0$ again (see (c)). These first three outcomes consume the first three tokens in Alice's stream and Bob's stream. It consumes the first token in Carmen's stream.}
        \label{fig:abc-stream}
    \end{figure}
    

    For more general $t$, these simple behavior streams are replaced with more complicated behaviors. For example, for $t = \lott$ Alice's behavior may be that of a biased coin whose bias may change over time. 
\end{example}

Distilling this example, a morphism $\org_\free\op(\yon^{A_1}, \cdots, \yon^{A_n}; \yon^B)$ is a  dynamic pattern that defines how to turn streams of tokens in $A_1, \cdots, A_n$ into a stream of tokens in $B$. Therefore we defined the category of \defined{stream processors} to be the full subcategory of $\org_\free\op$ spanned by monomials.

\newcommand{\huff}{{\mathsf{huff}}}

\begin{example}[Huffman coding]

Huffman coding is a technique for data compression in which more frequent symbols get shorter codes, while less frequent ones get longer codes. Given a distribution on $N$ symbols, its Huffman code is a binary code for each symbol. ~\cref{fig:huffman-standard} exemplifies how to produce a Huffman code for a language with symbols and respective probabilities:

\begin{table}[h]
    \centering
    \begin{tabular}{|r|c|c|c|c|c|}\hline
        symbol & A & B &C &D &E \\ \hline
        probability of occurrence & 0.30 & 0.02 & 0.45 & 0.15 & 0.08 \\ \hline
    \end{tabular}
    \label{tab:symbols}
\end{table}


\begin{figure}
    \centering
    \begin{forest}
    for tree={circle,draw, l sep=15pt, s sep=40pt, grow'=north}
    [1.00
        [0.55, edge label={node[midway,left] {0}}
            [0.30, edge label={node[midway,left] {0}}, label=above:A]
            [0.25, edge label={node[midway,right] {1}}
                [0.1, edge label={node[midway,left] {0}}
                    [0.02, edge label={node[midway,left] {0}}, label=above:B]
                    [0.08, edge label={node[midway,right] {1}}, label=above:E]
                ]
                [0.15, edge label={node[midway,right] {1}}, label=above:D]
            ]
        ]
        [0.45, edge label={node[midway,right] {1}}, label=above:C]
    ]
\end{forest}

    \caption{A tree that defines the Huffman code for the data given in Table~\ref{tab:symbols}. The basic strategy of the Huffman code is to produce a binary tree by iteratively combining the two groups of symbols with lowest probability. For each symbol, the path from the root to that symbol's leaf defines a binary encoding of the symbol. In this example, the codes are $C$: 0, $A$: 10, $D$: 110, $E$: 1110, and $B$: 1111. Note that the most frequent symbol ($C$) has the shortest code while the least frequent symbols ($B$ and $E$) have the longest codes.}
    \label{fig:huffman-standard}
\end{figure}

Huffman coding corresponds to a polynomial functor \[
    \huff \colon \lott \to \free_{\yon^2}
\] which we define inductively as polynomial functors $\huff_N \colon \sum_{\Delta_N} \yon^N \to \free_{\yon^2}$ for $N \geq 1$. Note that the positions of $\free_{\yon^2}$ are binary trees whose directions are the leaves of the tree.

For the base case, $\huff_1 \colon \yon \to \free_{\yon^2}$ is the inclusion of $\yon$ into $\free_{\yon^2}$. For the inductive step, suppose we have defined $\huff_{N}$. Consider the cartesian map \[
    \sum_{\Delta_{N+1}} \yon^{N+1} \to \sum_{\Delta_N} \yon^N \tri (\yon + \yon^2)
\] defined as follows. On positions we map a distribution on $N+1$ tickets to a distribution on $N$ tickets by combining the two tickets with lowest probability. Each of the $N$ tickets maps to $\yon^2$ if it combines two tickets in the original distribution and to $\yon$ otherwise. On directions, it maps the two directions of $\yon^2$ to the two tickets that were combined. For the remaining tickets in $N$, it maps the single the direction of $\yon$ to the original ticket in $N+1$.\footnote{
    To give an unambiguous definition of this map, requires several design choices which we enumerate here by defining the map explicitly. On positions, let $P$ be a distribution on $N + 1$ tickets. Let $i_1 \leq i_2 \in 1, \cdots , N$ be the pair of tickets that minimizes $P(i_1) + P(i_2)$, $i_1$, and $i_1 + i_2$. This pair is unique. Then define $P'$ to be a distribution on $N$ tickets defined by \[
        P'(i) = \begin{cases}
            P(i_1) + P(i_2) & i = i_1\\
            P(i) & i < i_2, i \neq i_1 \\
            P(i-1) & i \geq i_2.
        \end{cases}
    \] Finally, to complete the map on positions, we define a map $N \to (\yon + \yon^2)(1)$ which takes $i$ to the $\yon^2$ position if $i = i_1$ and the $\yon$ position otherwise. On directions, it is defined by $(i_1, k) \mapsto i_k$ for $k = 1,2$, $(i, *) \mapsto i$ for $i < i_2, i \neq i_1$, and $(i, *) \mapsto i + 1$ for $i \geq i_2$.
}
Figure~\ref{fig:huffN} is an example of this map for $N = 4$ on the position $(0.30, 0.02, 0.45, 0.15, 0.08)$.


\begin{figure}
    \centering
    \[
\begin{tikzpicture}[trees, level distance=1.5cm,
  level 1/.style={sibling distance=10mm},
  level 2/.style={sibling distance=7.5mm},
  level 3/.style={sibling distance=1.25mm}]
\node (a) {$(0.3, 0.02, 0.45, 0.15, 0.08)$}
        child {node (a1) {}}
        child {node (a2) {}}
        child {node (a3) {}}
        child {node (a4) {}}
        child {node (a5) {}}
    ;

    \node[right = 1.5in of a] (b) {$(0.3, 0.1, 0.45, 0.15)$}
        child {
                child {node (b1) {}}
            }
        child {
                child {node (b2) {}}
                child {node (b3) {}}
            }
        child {
                child {node (b4) {}}
            }
        child {
                child {node (b5) {}}
            }
    ;
    \draw[|->, red, shorten >=5pt, shorten <=5pt] (a) to[out=0, in=180] (b);
    \draw[|->, red, shorten >=5pt, shorten <=5pt] (b1) to[out=90, in=90] (a1);
    \draw[|->, red, shorten >=5pt, shorten <=5pt] (b2) to[out=90, in=90] (a2);
    \draw[|->, red, shorten >=5pt, shorten <=5pt] (b3) to[out=90, in=90] (a5);
    \draw[|->, red, shorten >=5pt, shorten <=5pt] (b4) to[out=90, in=90] (a3);
    \draw[|->, red, shorten >=5pt, shorten <=5pt] (b5) to[out=90, in=90] (a4);

\end{tikzpicture}
\]
    \caption{An example of the map $\sum_{\Delta_{N+1}} \yon^{N+1} \to \sum_{\Delta_N} \yon^N \tri (\yon + \yon^2)$ for $N = 4$ on the position $(0.3, 0.02, 0.45, 0.15, 0.08)$. The red arrows depict the map on positions and directions. }
    \label{fig:huffN}
\end{figure}


Then we define $\huff_{N+1}$ to be the composite
\[
    \sum_{\Delta_{N+1}} \yon^{N+1} \to \sum_{\Delta_N}\yon^N \tri (\yon + \yon^2) \to \free_{\yon^2} \tri (\yon + \yon^2) \to \free_{\yon^2} \tri \free_{\yon^2} \to \free_{\yon^2}
\] where the second to last map is given by the inclusion $\yon + \yon^2 \to \free_{\yon^2}$ and the final map is the $\tri$-monoidal structure on $\free_{\yon^2}$.
Figure~\ref{fig:huff} gives an example of $\huff$ on the position $(0.3, 0.02, 0.45, 0.15, 0.08)$.

\begin{figure}
    \centering
    \[
\begin{tikzpicture}[trees, level distance=1cm,
  level 1/.style={sibling distance=10mm}]
\node (a) {$(0.3, 0.02, 0.45, 0.15, 0.08)$}
        child {node (a1) {}}
        child {node (a2) {}}
        child {node (a3) {}}
        child {node (a4) {}}
        child {node (a5) {}}
    ;

    \node[right = 1.5in of a] (b) {}
        child {
            child {node (b1) {}}
                child {
                    child {
                        child {node (b2) {}}
                        child {node (b3) {}}
                    }
                    child {node (b4) {}}
                }
        }
        child {node (b5) {}}
    ;
    
    \draw[|->, red, shorten >=5pt, shorten <=5pt] (a) to[out=0, in=180] (b);
    \draw[|->, red, shorten >=5pt, shorten <=5pt] (b1) to[out=90, in=90] (a1);
    \draw[|->, red, shorten >=5pt, shorten <=5pt] (b2) to[out=90, in=90] (a2);
    \draw[|->, red, shorten >=5pt, shorten <=5pt] (b5) to[out=90, in=90] (a5);
    \draw[|->, red, shorten >=5pt, shorten <=5pt] (b4) to[out=90, in=90] (a3);
    \draw[|->, red, shorten >=5pt, shorten <=5pt] (b3) to[out=90, in=90] (a4);

\end{tikzpicture}
\]
    \caption{The action of $\huff \colon \lott \to \free_{\yon^2}$ on the position $(0.3, 0.02, 0.45, 0.15, 0.08)$.}
    \label{fig:huff}
\end{figure}

The map $\huff$ --- equivalent to a map $\yon \to [\lott, \free_{\yon^2}]$ --- defines the \emph{pattern} of the Huffman code for every lottery. The pattern-to-matter operation given by the functor in Theorem~\ref{thm:matter-pattern} turns this pattern into a dynamic decoder as follows. 

Let $A$ be an alphabet. A position of $\lott \tri A$ is equivalent to a choice of lottery and for each ticket a symbol in $A$.  Together with the map $\huff$, these  define a map:
\begin{align*}
    \yon & \to (\lott \tri A\yon) \otimes [\lott, \free_{\yon^2}] & \\
    & \to (\lott \tri A\yon) \otimes [\cofree_{2\yon}, [\lott, \yon]] & \\
    & \to [\cofree_{2\yon}, \yon ] \tri A\yon & \text{duoidality and evaluation}\\
    & \to [\cofree_{2\yon}, A\yon] & \text{duodality}
\end{align*}
This map is equivalent to a polynomial map $\cofree_{2\yon} \to A\yon$. On positions it maps a stream of $0$s and $1$s to the symbol of $A$ whose code uniquely begins the stream. On directions, it selects the leading bits that are part of the code. 

By iteratively decoding, this polynomial map is a retrofunctor $\cofree_{2\yon} \to \cofree_{A\yon}$. The positions of $\cofree_{2\yon}$ are bit streams and for each position it has $\nn$ directions which represent the number of initial bits produced/consumed. Likewise, the positions of $\cofree_{A\yon}$ are streams of symbols in $A$ and for each position it has $\nn$ directions which represent the number of initial symbols produced/consumed. 

On positions the induced retrofunctor $\cofree_{2\yon} \to \cofree_{A\yon}$ decodes a bit stream into a stream of symbols in $A$. On directions, it maps $n$ symbols produced to the number of initial bits that encode the $n$ symbols.

For example consider the position of $\lott \tri \{A, B, C, D, E\}$ given by the lottery on $5$ tickets with probabilities $(0.3, 0.02, 0.45, 0.15, 0.08)$ and where each of the 5 tickets is mapped to $A$, $B$, $C$, $D$, and $E$ respectively. This data defines a retrofunctor $\cofree_{2\yon} \to \cofree_{\{A, B, C, D, E\}\yon}$. An example of the behavior of this retrofunctor is give in Figure~\ref{fig:huff-stream}.


\begin{figure}
        \begin{tabular}{cc| c| ccc|cc|c|cccc|cc}
         \texttt{0} & \texttt{0} & \texttt{1} & \texttt{0} & \texttt{1} & \texttt{1} & \texttt{0} & \texttt{0} & \texttt{1} & \texttt{0} & \texttt{1} & \texttt{0} & \texttt{0} & \texttt{0} & $\cdots$
    \end{tabular}
    
    \vspace{10pt}
    
    \begin{tabular}{cc c cccccccccccc}
         \texttt{A} & \texttt{C} & \texttt{D} & \texttt{A} & \texttt{C} & \texttt{B} & $\cdots$ &  &  &  &  & &  & & 
    \end{tabular}
    \caption{On positions, a bit stream beginning with \texttt{00101100101000...} produces a stream of symbols beginning with \texttt{ACDACB...}. In the top line the $|$ depicts the action on directions. In particular, the direction corresponding to the prefix \texttt{A}, maps to the prefix \texttt{00}, since this is the Huffman encoding of the symbol $A$. It maps the direction corresponding to the prefix \texttt{AC} to the prefix \texttt{001}. And so forth.}
    \label{fig:huff-stream}
\end{figure}


\end{example}
