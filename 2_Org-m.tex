\chapter{An operad of pattern delegation}\label{sec:orgm}

In this section we will promote the free monad monad $\free$ on $\poly$ to a monad on $\org^\#$ and show that it is lax monoidal with respect to the monoidal product $\vee$. Then the Kleisli construction defines an $\smcat^\#$-enriched operad $\org_\free^\#$.

\section{Extension of $\free$ to a monad on $\org^\sharp$}

We will define a $\smcat^\sharp$-enriched functor $\free \colon \org^\sharp \to \org^\sharp$ and show that it is a monad. 

On objects $\free$ takes a polynomial $p$ to the free monad $\free_p$. On hom-objects, $\free \colon \cofree_{[p, q]} \to \cofree_{[\free_p, \free_q]}$ can be identified with the image of the composite, 
\[
    \free_p \otimes \cofree_{[p,q]} \xrightarrow{\modstruct_{p, [p, q]}} \free_{p \otimes [p, q]} \xrightarrow{\free_\eval} \free_q
\] where the identification is given by the following isomorphism:
\[
    \poly(\free_p\otimes \cofree_{[p, q]}, \free_q) \iso \poly(\cofree_{[p,q]}, [\free_p, \free_q]) \iso \smcat^\sharp(\cofree_{[p,q]}, \cofree_{[\free_p, \free_q]}).
\]
Here, the first isomorphism is the $\otimes$-hom adjunction and the second isomorphism is induced by the cofree adjunction \eqref{eqn.cofree_adjunction}.

We first check that $\free$ is functorial. It preserves identity because the composite
\[
    p \to p \otimes [p,p] \xrightarrow{\eval} p
\] is the identity on $p$, where the first map is the interal-hom identity. It preserves composition because the following diagram commutes % https://q.uiver.app/#q=WzAsNCxbMCwwLCJwIFxcb3RpbWVzIFtwLHFdIFxcb3RpbWVzIFtxLCByXSJdLFswLDEsInEgXFxvdGltZXMgW3Escl0iXSxbMSwxLCJyIl0sWzEsMCwicCBcXG90aW1lcyBbcCwgcl0iXSxbMCwxXSxbMSwyXSxbMywyXSxbMCwzXV0=
\[\begin{tikzcd}
	{p \otimes [p,q] \otimes [q, r]} & {p \otimes [p, r]} \\
	{q \otimes [q,r]} & r
	\arrow[from=1-1, to=1-2]
	\arrow[from=1-1, to=2-1]
	\arrow[from=1-2, to=2-2]
	\arrow[from=2-1, to=2-2]
\end{tikzcd}\]
The top horizontal map is given by internal-hom composition.


\begin{theorem}
    The $\smcat^\sharp$-enriched functor $\free \colon \org^\sharp \to \org^\sharp$ is a monad. 
\end{theorem}
\begin{proof}
    We need to define $\smcat^\sharp$-enriched natural transformations $\id_\poly \Rightarrow \free$ and $\free \circ \free \Rightarrow \free$ for the unit and multiplication.

    For a polynomial $p$, the identity at $p$ is an element of $\smcat^\sharp(\yon, \cofree_{[p, \free_p]})$. We define it to be the image of $\mun_p \colon p \to \free_p$  under the isomorphims
    \[
        \poly(p, \free_p) \iso \poly(\yon, [p, \free_p]) \iso \smcat^\sharp(\yon, \cofree_{[p, \free_p]}).
    \]

    To show that the identity is natural, we must show that for all polynomials $p$ and $q$, the following diagram of retrofunctors commutes. 
    % https://q.uiver.app/#q=WzAsNixbMCwwLCJcXGNvZnJlZV97W3AscV19Il0sWzEsMF0sWzIsMCwiXFxjb2ZyZWVfe1twLHFdfSBcXG90aW1lcyBcXGNvZnJlZV97W3EsIFxcZnJlZV9xXX0iXSxbMCwxLCJcXGNvZnJlZV97W1xcZnJlZV9wLCBcXGZyZWVfcV19Il0sWzEsMSwiXFxjb2ZyZWVfe1twLCBcXGZyZWVfcF19IFxcb3RpbWVzIFxcY29mcmVlX3tbXFxmcmVlX3AsIFxcZnJlZV9xXX0iXSxbMiwxLCJcXGNvZnJlZV97W3AsIFxcZnJlZV9xXX0iXSxbMCwyXSxbMCwzXSxbMyw0XSxbNCw1XSxbMiw1XV0=
    \[\begin{tikzcd}
    {\cofree_{[p,q]}} & {} & {\cofree_{[p,q]} \otimes \cofree_{[q, \free_q]}} \\
    {\cofree_{[\free_p, \free_q]}} & {\cofree_{[p, \free_p]} \otimes \cofree_{[\free_p, \free_q]}} & {\cofree_{[p, \free_q]}}
    \arrow[from=1-1, to=1-3]
    \arrow[from=1-1, to=2-1]
    \arrow[from=1-3, to=2-3]
    \arrow[from=2-1, to=2-2]
    \arrow[from=2-2, to=2-3]
    \end{tikzcd}\]

    By the universal property of retrofunctor maps into the cofree comonad $\cofree_{[p, \free_q]}$, it suffices to show that the following diagram commutes.
    % https://q.uiver.app/#q=WzAsNixbMCwwLCJwIFxcb3RpbWVzIFxcY29mcmVlX3tbcCxxXX0iXSxbMCwxLCJcXGZyZWVfcCBcXG90aW1lcyBcXGNvZnJlZV97W3AscV19Il0sWzEsMCwicCBcXG90aW1lcyBbcCwgcV0iXSxbMSwxLCJcXGZyZWVfe3AgXFxvdGltZXMgW3AscV19Il0sWzIsMSwiXFxmcmVlX3EiXSxbMiwwLCJxIl0sWzAsMiwicCBcXG90aW1lcyBcXGNjb3VuX3tbcCxxXX0iXSxbMiw1LCJcXGV2YWwiXSxbNSw0LCJcXG11bl9xIl0sWzIsMywiXFxtdW5fe3AgXFxvdGltZXMgW3AscV19Il0sWzMsNCwiXFxmcmVlX1xcZXZhbCIsMl0sWzEsMywiXFxtb2RzdHJ1Y3Rfe3AsIFtwLHFdfSIsMl0sWzAsMSwiXFxtdW5fcCBcXG90aW1lcyBcXGNvZnJlZV97W3AscV19IiwyXV0=
    \[\begin{tikzcd}[column sep=huge, row sep = large]
	{p \otimes \cofree_{[p,q]}} & {p \otimes [p, q]} & q \\
	{\free_p \otimes \cofree_{[p,q]}} & {\free_{p \otimes [p,q]}} & {\free_q}
	\arrow["{p \otimes \ccoun_{[p,q]}}", from=1-1, to=1-2]
	\arrow["{\mun_p \otimes \cofree_{[p,q]}}"', from=1-1, to=2-1]
	\arrow["\eval", from=1-2, to=1-3]
	\arrow["{\mun_{p \otimes [p,q]}}", from=1-2, to=2-2]
	\arrow["{\mun_q}", from=1-3, to=2-3]
	\arrow["{\modstruct_{p, [p,q]}}"', from=2-1, to=2-2]
	\arrow["{\free_\eval}"', from=2-2, to=2-3]
    \end{tikzcd}\]
    The left-hand square commute by definition of the interaction law (see \cite[Section 3.2]{libkind2024pattern}) and the right-hand square commutes by naturality of of $\mun$.

    For a polynomial $p$, define the multiplication at $p$ to be the image of the counit $\mcoun_{\free_p} \colon \free_{\free_p} \to \free_p$ under the isomorphisms
    \[
        \poly(\free_{\free_p}, \free_p) \iso \poly(\yon, [\free_{\free_p}, \free_p]) \iso \smcat^\sharp(\yon, \cofree_{[\free_{\free_p}, \free_p]}).
    \] 
    To show that the multiplication is natural we must show that for all polynomial $p$ and $q$, the following diagram of retrofunctors commutes. 

    % https://q.uiver.app/#q=WzAsNyxbMSwwLCJcXGNvZnJlZV97W1xcZnJlZV9wLFxcZnJlZV9xXX0iXSxbMiwwLCJcXGNvZnJlZV97W1xcZnJlZV97XFxmcmVlX3B9LCBcXGZyZWVfe1xcZnJlZV9xfV19Il0sWzMsMCwiXFxjb2ZyZWVfe1tcXGZyZWVfe1xcZnJlZV9wfSwgXFxmcmVlX3tcXGZyZWVfcX1dfSBcXG90aW1lcyBcXGNvZnJlZV97W1xcZnJlZV97XFxmcmVlX3F9LCBcXGZyZWVfcV19Il0sWzIsMSwiXFxjb2ZyZWVfe1tcXGZyZWVfe1xcZnJlZV9wfSwgXFxmcmVlX3BdfVxcb3RpbWVzIFxcY29mcmVlX3tbXFxmcmVlX3AsIFxcZnJlZV9xXX0iXSxbMywxLCJcXGNvZnJlZV97W1xcZnJlZV97XFxmcmVlX3B9LCBcXGZyZWVfcV19Il0sWzAsMCwiXFxjb2ZyZWVfe1twLHFdfSJdLFswLDEsIlxcY29mcmVlX3tbXFxmcmVlX3AsIFxcZnJlZV9xXX0iXSxbMCwxXSxbMSwyXSxbMiw0XSxbMyw0XSxbNiwzXSxbNSw2XSxbNSwwXV0=
      \[\begin{tikzcd}
    {\cofree_{[p,q]}} & {\cofree_{[\free_p,\free_q]}} & {\cofree_{[\free_{\free_p}, \free_{\free_q}]}} & {\cofree_{[\free_{\free_p}, \free_{\free_q}]} \otimes \cofree_{[\free_{\free_q}, \free_q]}} \\
    {\cofree_{[\free_p, \free_q]}} && {\cofree_{[\free_{\free_p}, \free_p]}\otimes \cofree_{[\free_p, \free_q]}} & {\cofree_{[\free_{\free_p}, \free_q]}}
    \arrow[from=1-1, to=1-2]
    \arrow[from=1-1, to=2-1]
    \arrow[from=1-2, to=1-3]
    \arrow[from=1-3, to=1-4]
    \arrow[from=1-4, to=2-4]
    \arrow[from=2-1, to=2-3]
    \arrow[from=2-3, to=2-4]
  \end{tikzcd}\]

  Again by the universal property of the cofree comonad $\cofree_{[\free_{\free_p}, \free_q]}$ it suffices to show that the following diagram commutes.
  % https://q.uiver.app/#q=WzAsNixbMCwwLCJcXGZyZWVfe1xcZnJlZV9wfSBcXG90aW1lcyBcXGNvZnJlZV97W1xcZnJlZV9wLCBcXGZyZWVfcV19Il0sWzAsMSwiXFxmcmVlX3AgXFxvdGltZXMgXFxjb2ZyZWVfe1tcXGZyZWVfcCwgXFxmcmVlX3FdfSJdLFsxLDEsIlxcZnJlZV9wIFxcb3RpbWVzIFtcXGZyZWVfcCwgXFxmcmVlX3FdIl0sWzEsMCwiXFxmcmVlX3tcXGZyZWVfcCBcXG90aW1lcyBbXFxmcmVlX3AsIFxcZnJlZV9xXX0iXSxbMiwxLCJcXGZyZWVfcSJdLFsyLDAsIlxcZnJlZV97XFxmcmVlX3F9Il0sWzAsMSwiXFxtY291bl97XFxmcmVlX3B9IFxcb3RpbWVzIFxcY29mcmVlX3tbXFxmcmVlX3AsIFxcZnJlZV9xXX0iLDJdLFsxLDIsIlxcZnJlZV9wIFxcb3RpbWVzIFxcY2NvdW5fe1tcXGZyZWVfcCwgXFxmcmVlX3FdfSIsMl0sWzAsMywiXFxtb2RzdHJ1Y3Rfe1xcZnJlZV9wLCBbXFxmcmVlX3AsIFxcZnJlZV9xXX0iXSxbMiwzLCJcXG11bl97XFxmcmVlX3AgXFxvdGltZXMgW1xcZnJlZV9wLCBcXGZyZWVfcV19Il0sWzMsNSwiXFxmcmVlX3tcXGV2YWx9Il0sWzUsNCwiXFxtY291bl97XFxmcmVlX3F9Il0sWzIsNCwiXFxldmFsIiwyXV0=
\[\begin{tikzcd}[column sep=huge,row sep=25pt]
	{\free_{\free_p} \otimes \cofree_{[\free_p, \free_q]}} & {\free_{\free_p \otimes [\free_p, \free_q]}} & {\free_{\free_q}} \\
	{\free_p \otimes \cofree_{[\free_p, \free_q]}} & {\free_p \otimes [\free_p, \free_q]} & {\free_q}
	\arrow["{\modstruct_{\free_p, [\free_p, \free_q]}}", from=1-1, to=1-2]
	\arrow["{\mcoun_{\free_p} \otimes \cofree_{[\free_p, \free_q]}}"', from=1-1, to=2-1]
	\arrow["{\free_{\eval}}", from=1-2, to=1-3]
	\arrow["{\mcoun_{\free_q}}", from=1-3, to=2-3]
	\arrow["{\free_p \otimes \ccoun_{[\free_p, \free_q]}}"', from=2-1, to=2-2]
	\arrow["{\mun_{\free_p \otimes [\free_p, \free_q]}}", from=2-2, to=1-2]
	\arrow["\eval"', from=2-2, to=2-3]
\end{tikzcd}\]

    The left-hand square commutes by definition of the interaction law $\modstruct$ and the right-hand square commutes by the zig-zag laws of the adjunction.

    These maps satisfy the identity law because of the zig-zag laws of the adjunction. Multiplication satisfies associativity by naturality of the counit.
\end{proof}

\begin{remark}\label{rmk:free-lifts}
    The composite $U \circ \free \colon \poly \to \poly$ is a monad. 
    How is it related to our monad $\free \colon \org^\sharp \to \org^\sharp$? 
    Consider the map of enriching functors $\smcat^\sharp(\yon , -) \colon \smcat^\sharp \to \smset$. Under $\smcat^\sharp(\yon , -)$, the morphisms $\cofree_{[p,q]}$ in $\org^\sharp$ map to 
    \[
        \smcat^\sharp(\yon, \cofree_{[p,q]}) \iso \poly(\yon, [p,q]) \iso \poly(p, q).
    \]
    Therefore, this map of enriching functors sends $\org^\sharp$ to $\poly$ and sends the monad $\free \colon \org^\sharp \to \org^\sharp$ to the monad $U \circ \free \colon \poly \to \poly$.
\end{remark}

\begin{remark}
The argument of this section works for any monad $M\colon\poly\to\poly$ equipped with a $\cofree_-$-module structure \cite{nlab:module_over_a_monoidal_functor}, i.e.\ a natural map
\[
	M(p)\otimes\cofree_q\to M(p\otimes q)
\]
satisfying the action laws for $p\mapsto\cofree_p$ as a $\otimes$-lax monoidal functor. That is, any such monad lifts to a monad $M\colon\org^\sharp\to\org^\sharp$.

For example, if $(t,\eta,\mu)$ is any polynomial monad, such as the list monad, then $p\mapsto p\tri t$ and $p\mapsto t\tri p$ are monads on $\poly$ that can be given $\cofree$-module structures, and hence lift to monads on $\org^\sharp$. In fact, such monads distribute over the free monad, so that both $\free_{-\tri t}$ and $\free_{t\tri-}$ are monads on $\poly$, and they again can be equipped with a $\cofree_-$-module structure. Hence for any polynomial monad $t$ we have monads
\[
	(p\mapsto\free_{p\tri t})\colon
	\org^\sharp\to\org^\sharp
	\qqand
	(p\mapsto\free_{t\tri p})\colon
	\org^\sharp\to\org^\sharp.
	\qedhere
\]
\end{remark}


\section{The operad $\org^\sharp_\free$}\label{sec:operad-orgm}

We begin by showing that $\org^\sharp$ has a symmetric monoidal structure $(\vee, 0)$ and that $\free \colon \org^\sharp \to \org^\sharp$ is lax monoidal with respect to $\vee$. The associated Kleisli category $\org^\#_\free$ is monoidal and hence has an underlying operad which we also refer to as $\org^\#_\free$.

Before discussing the monoidal structure, we know from the previous section that $(p\mapsto\free_p)\colon\org^\sharp\to\org^\sharp$ is a  $(\catsharp,\yon,\otimes)$-enriched monad, so its Kleisli category has polynomials as objects and hom-objects of the form
\begin{equation}
	\ul{\Hom}(p,q)\coloneqq\cofree_{[p,\free_q]}.
\end{equation}
The retrofunctor $\ul{\Hom}(p,q)\otimes\ul{\Hom}(q,r)\to\ul{\Hom}(p,r)$ is equivalently a polynomial map $p\otimes\cofree_{[p,\free_q]}\otimes\cofree_{[q,\free_r]}\to\free_r$, which is obtained as follows
\begin{equation}
  p\otimes\cofree_{[p,\free_q]}\otimes\cofree_{[q,\free_r]}\To{\ccoun_{[p, \free_q]}}
  p\otimes[p,\free_q]\otimes\cofree_{[p,\free_q]}\To{\text{eval}}
  \free_q\otimes\cofree_{[q,\free_r]}\To{\Xi}
  \free_{q\otimes\free_{[q,\free_r]}}\To{\text{eval}}
  \free_{\free_r}\to\free_r.
\end{equation}

\begin{example}\label{ex.sequence_counter}
The $\catsharp$-enriched monoid spanned by the polynomial $\yon$ can be thought of as the theory of step-counting. It has a single object, $\yon$, whose category of endomorphisms is
\[
\ul\Hom(\yon,\yon)=\cofree_{[\yon,\free_\yon]}=\cofree_{\nn\yon}=\nn^\nn\yon^\nn.
\]
This is the free category on the graph whose set of vertices is $\nn^\nn$, the set of natural number sequences (streams), and for which an edge is given by taking the tail. 

The monoidal product $\nn^\nn\yon^\nn\otimes\nn^\nn\yon^\nn\to\nn^\nn\yon^\nn$ is a retrofunctor, which we denote $\star$. Even as a monoid operation $\star\colon\nn^\nn\times\nn^\nn\to\nn$ on the set of natural number sequences, it appears to be novel. The unit is $(1,1,\ldots)$ and for sequences $A,B\colon\nn\to\nn$ the monoidal product is given as follows. First, for any $n\in\nn$, let $a_n\coloneqq\sum_{i=0}^nA_n$, so $a_0=A_0$, $a_1=A_0+A_1$, etc., and for convenience, define $a_{-1}\coloneqq 0$; so $a_n$ is defined for $n:\nn\cup\{-1\}$. Then for any $n\in\nn$, define
\[
(A\star B)_n\coloneqq\sum_{i=a_{n-1}}^{a_n-1}B_i
\]
So for example, $(0,1,2,3,4,\ldots)\star(0,1,2,3,4,\ldots)=(0,0,3,12,30,60,\ldots)$. The each step (position) in the first list indicates how many steps to perform in the second, and this is an associative operation.
\end{example}

We now turn to the monoidal structure on $\org_\free$. It begins with the symmetric monoidal structure $\vee$ on $\poly$ whose unit is $0$ and whose product is defined by 
\begin{equation}\label{eqn.vee}
    p \vee q \coloneqq p + (p \otimes q) + q.
\end{equation}


\begin{lemma}
    The functor $U \circ \free \colon \poly \to \poly$ is lax monoidal with respect to $\vee$.
\end{lemma}
\begin{proof}
    First we show by transfinite induction that there is a map $\free_p \to [ \free_{q}, \free_{p \vee q}]$. Recall the construction of $\free_p=\colim(\cdots \to p_{(\alpha)}\To{\iota_\alpha}p_{(\alpha+1)}\to\cdots)$ as in \cref{sec.(co)free(co)monad}. It suffices to show that for each ordinal $\alpha$ there is a map $p\hoc{\alpha} \otimes \free_q \to \free_{p \vee q}$ such that the following commutes for all ordinals $\alpha$, 
    % https://q.uiver.app/#q=WzAsMyxbMCwwLCJwXFxob2N7XFxhbHBoYX1cXG90aW1lcyBcXGZyZWVfcSJdLFswLDEsInBcXGhvY3tcXGFscGhhKzF9IFxcb3RpbWVzIFxcZnJlZV9xIl0sWzEsMSwiXFxmcmVlX3twIFxcdmVlIHF9Il0sWzAsMSwiXFxpb3RhIFxcaG9je1xcYWxwaGF9IiwyXSxbMCwyXSxbMSwyXV0=
\begin{equation}\label{eq:p_freeq_freepveeq}
\begin{tikzcd}
	{p\hoc{\alpha}\otimes \free_q} \\
	{p\hoc{\alpha+1} \otimes \free_q} & {\free_{p \vee q}}
	\arrow["{\iota \hoc{\alpha}}"', from=1-1, to=2-1]
	\arrow[from=1-1, to=2-2]
	\arrow[from=2-1, to=2-2]
\end{tikzcd}    
\end{equation}
First we define the maps.
    \begin{itemize}
        \item Base case. We define the map $p\hoc{0} \otimes \free_q = \yon \otimes \free_q \to  \free_{p \vee q}$ to be induced by $\free$ applied to the inclusion $q \to p \vee q$.

        \item For successor ordinals $\alpha +1$,
        \begin{align*}
            p\hoc{\alpha + 1} \otimes \free_q  & = (\yon + p \tri p\hoc{\alpha})\otimes \free_ q \\
            & \to \free_q  + p \tri(p \hoc{\alpha} \otimes \free_q) \\
            & \to \free_q + p \tri \free_{p \vee q}\\
            & \to \free_{p \vee q} + (p \vee q) \tri \free_{p \vee q}\\
            & \iso \free_{p \vee q}.
        \end{align*}
        follows by distributivity of $\otimes$, by duoidality, and by the induction hypothesis. 

        That these maps are natural with respect to the inclusions $\iota\hoc{\alpha} \colon p\hoc{\alpha} \to p \hoc{\alpha + 1}$ is straightforward by induction.
        
        \item For limit ordinals $\alpha$, define 
        $\left(\colim_{\alpha' < \alpha} p\hoc{\alpha'}\right) \otimes \free_q \to \free_{p \vee q}$  to be the universal map induced by the components $p\hoc{\alpha'} \otimes \free_q \to \free_{p \vee q}$ given by the induction hypothesis. 
\end{itemize}
        Lastly we must show that the diagram in \cref{eq:p_freeq_freepveeq} commutes. By definition of $\iota\hoc{\alpha}$ it suffices to show that for all $\alpha' < \alpha$, the following diagram commutes. 

        % https://q.uiver.app/#q=WzAsNSxbMCwwLCIoXFx5b24gKyBwIFxcdHJpIHAgXFxob2N7XFxhbHBoYSd9KSBcXG90aW1lcyBcXGZyZWVfcSJdLFswLDEsIlxcbGVmdChcXHlvbiArIHAgXFx0cmkgIFxcbGVmdChcXGNvbGltX3tcXGFscGhhJyA8IFxcYWxwaGF9IHAgXFxob2N7XFxhbHBoYSd9IFxccmlnaHQpXFxyaWdodCkgXFxvdGltZXMgXFxmcmVlX3EiXSxbMSwxLCJcXGZyZWVfcSAgKyBwIFxcdHJpICgoXFxjb2xpbV97XFxhbHBoYScgPCBcXGFscGhhfSBwIFxcaG9je1xcYWxwaGEnfSApXFxvdGltZXMgXFxmcmVlX3EpIl0sWzEsMCwiXFxmcmVlX3EgKyBwIFxcdHJpIChwIFxcaG9je1xcYWxwaGEnfSBcXG90aW1lcyBcXGZyZWVfcSkiXSxbMiwxLCJcXGZyZWVfe3AgXFx2ZWUgcX0iXSxbMCwxXSxbMCwzXSxbMSwyXSxbMywyXSxbMiw0XSxbMyw0XV0=
\[\begin{tikzcd}
	{(\yon + p \tri p \hoc{\alpha'}) \otimes \free_q} & {\free_q + p \tri (p \hoc{\alpha'} \otimes \free_q)} \\
	{\left(\yon + p \tri  \left(\colim_{\alpha' < \alpha} p \hoc{\alpha'} \right)\right) \otimes \free_q} & {\free_q  + p \tri ((\colim_{\alpha' < \alpha} p \hoc{\alpha'} )\otimes \free_q)} & {\free_{p \vee q}}
	\arrow[from=1-1, to=1-2]
	\arrow[from=1-1, to=2-1]
	\arrow[from=1-2, to=2-2]
	\arrow[from=1-2, to=2-3]
	\arrow[from=2-1, to=2-2]
	\arrow[from=2-2, to=2-3]
\end{tikzcd}\]

    The left-hand square commutes by naturality of duoidality and the distributive law. The right-hand triangle commutes by definition of $\left(\colim_{\alpha' < \alpha} p \hoc{\alpha'}\right) \otimes \free_q \to \free_{p \vee q}$.


    We have now defined a map $\free_p \to [ \free_{q}, \free_{p \vee q}]$. The induced map $\free_p \otimes \free_q \to \free_{p \vee q}$ along with the maps $\free_p \to \free_{p \vee q}$ and $\free_q \to \free_{p \vee q}$ given by $\free$ applied to the inclusions $p \to p \vee q$ and $q \to p \vee q$, together define the compositor 
    \[
        \free_p \vee \free_q \to \free_{p \vee q}.
    \]
    The unitor is the unique map $\mun_0  \colon  0 \to \free_0$.

    Associativity is straightforward by transfinite induction and  unitality is immediate. 
\end{proof}


We can extend the symmetric monoidal structure $\vee$ to $\org^\sharp$ as follows.
The monoidal structure on objects is the same as that for $\poly$.
We define the monoidal structure on morphisms in three components. First, there is a map 
\[
    p \otimes \cofree_{[p,q]} \otimes \cofree_{[p', q']} \to p \otimes \cofree_{[p,q]} \xrightarrow{p \otimes \ccoun_{[p,q]}} p \otimes [p,q] \to q \to q \vee q'.
\] where the first map is induced by the $\tri$-comonoid counit $\cofree_{[p', q']}\to\yon$. Likewise there is a map $p' \otimes \cofree_{[p,q]} \otimes \cofree_{[p', q']} \to q \vee q'$. Lastly, there is a map 
\[
    p \otimes p' \otimes \cofree_{[p,q]} \otimes \cofree_{[p', q']} \xrightarrow{p \otimes p' \otimes \ccoun_{[p,q]} \otimes \ccoun_{[p', q']}} p \otimes p' \otimes [p,q] \otimes [p', q'] \to q \otimes q' \to q \vee q'.
\] 
By definition of $\vee$ as a coproduct \eqref{eqn.vee} and the distributivity of $\otimes$ over $+$, the above three maps induce a map
\[
    (p \vee p')\otimes \cofree_{[p,q]} \otimes \cofree_{[p', q']} \to q \vee q'.
\]
The action of the monoidal product on morphisms is the image of this map under the isomophisms,
\begin{align*}
    \poly((p \vee p')\otimes \cofree_{[p,q]}\otimes \cofree_{[p', q']}, q \vee q') &\iso \poly(\cofree_{[p,q]}\otimes \cofree_{[p', q']}, [p \vee p', q \vee q']) \\
    &\iso \smcat^\sharp(\cofree_{[p,q]}\otimes \cofree_{[p', q']}, \cofree_{ [p \vee p', q \vee q']}).
\end{align*}

The associator, left and right unitors, triangle identity, and pentagon identity all follow directly from the fact that $(0,\vee)$ is a symmetric monoidal structure on $\poly$ \cite{spivak2022reference}.


    

\begin{theorem}
    The $\smcat^\sharp$-enriched functor $\free \colon \org^\sharp \to \org^\sharp$ is lax monoidal with respect to the monoidal structure $\vee$.
\end{theorem}
\begin{proof}
    The unitor is the image of the unitor (which is the unique element) of $\free \colon \poly \to \poly$ under the isomorphisms
    \[
        \poly(0, \free_0) \iso \poly(\yon , [0, \free_0]) \iso \smcat^\sharp(\yon, \cofree_{[0, \free_0]}).
    \]
    The compositor is the image of the compositor of $\free \colon \poly \to \poly$ under the isomorphisms
    \[
        \poly(\free_p \vee \free_q, \free_{p \vee q}) \iso \poly(\yon, [\free_p \vee \free_q, \free_{p \vee q}]) \iso \smcat^\sharp(\yon, \cofree_{[\free_p \vee \free_q, \free_{p \vee q}]}).
    \]
    Associativity and unitality follow directly.
\end{proof}

\begin{definition}\label{def.orgsharp}
    Let $\org^\sharp_\free$ be the $\smcat^\sharp$-enriched operad underlying the Kleisli category $\org^\sharp_\free$. The objects of $\org^\sharp_\free$ are polynomials and the morphisms are defined by 
    \[
        \org^\sharp_\free(p_1 , \ldots, p_m; q) = \cofree_{[p_1 \vee \cdots \vee p_m, \free_q]}.
    \]

    Let $(\org^\sharp_\free)\op$ be the $\smcat^\sharp$-enriched operad underlying the opposite category $(\org^\sharp_\free)\op$. Its objects are polynomials and its morphisms are defined by 
    \[
        (\org^\sharp_\free)\op(q_1, \ldots, q_m; p) = \cofree_{[p, \free_{q_1 \vee \cdots \vee q_m}]}.
    \]

    Let $\org_\free$ and $\org_\free\op$ be the corresponding $\smcat$-enriched categories under the map of enriching categories $\smset^- \colon \smcat^\sharp \to \smcat$.
\end{definition}




In the $\smcat$-enriched category $\org_\free\op$, an object is a polynomial and a morphism in $\org_\free\op(p_1, \cdots, p_m; q)$ is a $[q, \free_{p_1 \vee \cdots \vee p_m}]$-coalgebra (and coalgebra morphisms are 2-cells). We can think of a coalgebra $S\yon^S\to[q, \free_{p_1 \vee \cdots \vee p_m}]$ as a dynamically changing strategy that completes $q$-shaped tasks by delegating to $p_i$-shaped subordinates. Given a position of $q$---which we think of as a $q$ task---the state of the coalgebra determines a decision tree made up of $p_i$ component tasks. At each fork of the decision tree one or more $p_i$ subordinates may be invoked. Each possible outcome either returns an outcome to the $q$ task and updates the state or it determines another set of $p_i$ tasks to be invoked. This can continue in an arbitrary number of iterations, but it is well-founded in the sense that it eventually terminates. 

Once an outcome is determined, the whole system can learn from the sequence of events. That is, the above story was that of a single state or map $q\To{s}\free_{p_1 \vee \cdots \vee p_m}$, i.e.\ a single position $\yon\To{s}[q, \free_{p_1 \vee \cdots \vee p_m}]$. The actual $q$-task assigned and the actual sequence of outcomes constitutes a direction of $[q, \free_{p_1 \vee \cdots \vee p_m}]$ at $s$, which updates the state. This is what we meant by ``dynamically changing strategy".

In the $\smcat$-enriched operad $\org$ defined in \cite{shapiro2022dynamic}, each subordinate must be consulted exactly once and the outcomes from each subordinate are aggregated into a single answer. The morphisms of $\org_\free\op$ are more general since subordinates may be consulted zero, one, or many times strategically as a result of previous subordinate answers. 

\begin{example}\label{ex:stream-m}

Consider the polynomials \textcolor{cyan}{$\alice = \yon^2$}, \textcolor{magenta}{$\bob = \yon^2$}, and \textcolor{teal}{$\carmen = \yon^2$}. These polynomials represent subordinates Alice, Bob, and Carmen who when consulted return one of two outcomes. 
A coalgebra in $\org_\free\op(\alice, \bob, \carmen; \yon^2)$ will consult Alice, Bob, and Carmen in order to produce one of two outcomes. It is a dynamically changing pattern for taking the task of determining one of two outcomes and delegating it to subordinates Alice, Bob, and Carmen.

To begin, consider the coalgebra with a single state that is defined by the polynomial map $\yon^2 \to \free_{\alice \vee \bob \vee \carmen}$ defined in \cref{fig:carmen-tie-breaks}. 

\begin{figure}[h!]
    \centering
    \[
\begin{tikzpicture}[trees,
   level distance=1.5cm,sibling distance=1.75cm, 
   edge from parent path={(\tikzparentnode) -- (\tikzchildnode)}]
\Tree [.$\bullet$
    \edge node[auto=left] {$(\azero, \bzero)$};  
    [.$\quad$ ]
    \edge node[auto=right] {$(\azero, \bone)$};
    [.$\bullet$
      \edge node[auto=left] {$\czero$};
      [.$\quad$ ] 
      \edge node[auto=right] {$\cone$};
      [.$\quad$ ] 
    ] 
    \edge node[auto=left] {$(\aone, \bzero)$};
    [.$\bullet$
      \edge node[auto=left] {$\czero$};
      [.$\quad$ ] 
      \edge node[auto=right] {$\cone$};
      [.$\quad$ ] 
    ] 
    \edge node[auto=right] {$(\aone, \bone)$};  
    [.$\quad$ ]
    ]
\end{tikzpicture}
\]
    \caption{A position of $\free_{\alice \vee \bob \vee \carmen}$. There is a delegation pattern $\yon^2 \to \free_{\alice \vee \bob \vee \carmen}$ that maps the single position of $\yon^2$ to this position $\free_{\alice \vee \bob \vee \carmen}$. It maps the directions from left to right to $0$, $0$, $1$, $0$, $1$, and $1$. }
    \label{fig:carmen-tie-breaks}
\end{figure}
 In other words, Alice and Bob are both asked for a $0$ or $1$. If their outcomes agree then that value is returned. Otherwise, Carmen is the tie-breaker. 

We can upgrade this coalgebra to have non-trivial dynamics. Consider a coalgebra with three states. The first state takes the single position of $\yon^2$ to the decision tree as above. The second state takes the single position of $\yon^2$ to a similar decision tree  that has Alice as the tie-breaker for Bob and Carmen. The third state takes the single position of $\yon^2$ to the decision tree where Bob is the tie-breaker for Alice and Carmen. On directions, if Carmen tie-breaks then the state updates so that Alice is the new tie-breaker. Once Alice tie-breaks, then the state updates so that Bob is the new tie-breaker. And so on.

A more complicated dynamics, involving a much larger state set, would be to learn a preference over initial consultants versus tie-breakers based on the quality of the results. Here, the state is some record of historical events and their quality scores.

Now imagining that the subordinates Alice, Bob, and Carmen also dynamically decide between $0$ and $1$ based on a decision tree of their own (sub-)subordinates, then composition in $\org_\free\op$ would defines a coalgebra which operates on these sub-subordinates. 


\end{example}