\chapter{$\org^\sharp_\free$}\label{sec:orgm}

\section{$\free$ is a monad on $\org^\sharp$}

We will define a $\smcat^\sharp$-enriched functor $\free \colon \org^\sharp \to \org^\sharp$ and show that it is a monad. 

On objects $\free$ takes a polynomial $p$ to the free monad $\free_p$. On morphisms $\free \colon \cofree_{[p, q]} \to \cofree_{[\free_p, \free_q]}$ is the image of the composite, 
\[
    \free_p \otimes \cofree_{[p,q]} \xrightarrow{\modstruct_{p, [p, q]}} \free_{p \otimes [p, q]} \xrightarrow{\free_\eval} \free_q
\] under the composite 
\[
    \poly(\free_p\otimes \cofree_{[p, q]}, \free_q) \iso \poly(\cofree_{[p,q]}, [\free_p, \free_q]) \iso \smcat^\sharp(\cofree_{[p,q]}, \cofree_{[\free_p, \free_q]}).
\]
where the first isomorphism is the $\otimes$-hom adjunction and the second isomorphism is induced by the fact that $\cofree_{[p,q]}$ is a comonad and $\cofree_{[\free_p, \free_q]}$ is cofree.

% Need to show that $\free$ satisfies preserves identity and composition

\begin{theorem}
    The $\smcat^\sharp$-enriched functor $\free \colon \org^\sharp \to \org^\sharp$ is a monad. 
\end{theorem}
\begin{proof}
    We need to define $\smcat^\sharp$-enriched natural transformations $\id_\poly \Rightarrow \free$ and $\free \circ \free \Rightarrow \free$ for the unit and multiplication.

    For a polynomial $p$, define the identity at $p$ is an element of $\smcat^\sharp(\yon, \cofree_{[p, \free_p]})$. We define it to be the image of $\mun_p \colon p \to \free_p$  under the isomorphims
    \[
        \poly(p, \free_p) \iso \poly(y, [p, \free_p]) \iso \smcat^\sharp(\yon, \cofree_{[p, \free_p]}.
    \]

    To show that the identity is natural, we must show that for all polynomials $p$ and $q$, the following diagram commutes. 
    % https://q.uiver.app/#q=WzAsNixbMCwwLCJcXGNvZnJlZV97W3AscV19Il0sWzEsMF0sWzIsMCwiXFxjb2ZyZWVfe1twLHFdfSBcXG90aW1lcyBcXGNvZnJlZV97W3EsIFxcZnJlZV9xXX0iXSxbMCwxLCJcXGNvZnJlZV97W1xcZnJlZV9wLCBcXGZyZWVfcV19Il0sWzEsMSwiXFxjb2ZyZWVfe1twLCBcXGZyZWVfcF19IFxcb3RpbWVzIFxcY29mcmVlX3tbXFxmcmVlX3AsIFxcZnJlZV9xXX0iXSxbMiwxLCJcXGNvZnJlZV97W3AsIFxcZnJlZV9xXX0iXSxbMCwyXSxbMCwzXSxbMyw0XSxbNCw1XSxbMiw1XV0=
    \[\begin{tikzcd}
    {\cofree_{[p,q]}} & {} & {\cofree_{[p,q]} \otimes \cofree_{[q, \free_q]}} \\
    {\cofree_{[\free_p, \free_q]}} & {\cofree_{[p, \free_p]} \otimes \cofree_{[\free_p, \free_q]}} & {\cofree_{[p, \free_q]}}
    \arrow[from=1-1, to=1-3]
    \arrow[from=1-1, to=2-1]
    \arrow[from=1-3, to=2-3]
    \arrow[from=2-1, to=2-2]
    \arrow[from=2-2, to=2-3]
    \end{tikzcd}\]

    It suffices to show that the following diagram commutes.
    % https://q.uiver.app/#q=WzAsNixbMCwwLCJwIFxcb3RpbWVzIFxcY29mcmVlX3tbcCxxXX0iXSxbMCwxLCJcXGZyZWVfcCBcXG90aW1lcyBcXGNvZnJlZV97W3AscV19Il0sWzEsMCwicCBcXG90aW1lcyBbcCwgcV0iXSxbMSwxLCJcXGZyZWVfe3AgXFxvdGltZXMgW3AscV19Il0sWzIsMSwiXFxmcmVlX3EiXSxbMiwwLCJxIl0sWzAsMiwicCBcXG90aW1lcyBcXGNjb3VuX3tbcCxxXX0iXSxbMiw1LCJcXGV2YWwiXSxbNSw0LCJcXG11bl9xIl0sWzIsMywiXFxtdW5fe3AgXFxvdGltZXMgW3AscV19Il0sWzMsNCwiXFxmcmVlX1xcZXZhbCIsMl0sWzEsMywiXFxtb2RzdHJ1Y3Rfe3AsIFtwLHFdfSIsMl0sWzAsMSwiXFxtdW5fcCBcXG90aW1lcyBcXGNvZnJlZV97W3AscV19IiwyXV0=
    \[\begin{tikzcd}[column sep=huge, row sep = large]
	{p \otimes \cofree_{[p,q]}} & {p \otimes [p, q]} & q \\
	{\free_p \otimes \cofree_{[p,q]}} & {\free_{p \otimes [p,q]}} & {\free_q}
	\arrow["{p \otimes \ccoun_{[p,q]}}", from=1-1, to=1-2]
	\arrow["{\mun_p \otimes \cofree_{[p,q]}}"', from=1-1, to=2-1]
	\arrow["\eval", from=1-2, to=1-3]
	\arrow["{\mun_{p \otimes [p,q]}}", from=1-2, to=2-2]
	\arrow["{\mun_q}", from=1-3, to=2-3]
	\arrow["{\modstruct_{p, [p,q]}}"', from=2-1, to=2-2]
	\arrow["{\free_\eval}"', from=2-2, to=2-3]
    \end{tikzcd}\]
    The left-hand square commute by definition of the interaction law and the right-hand square commutes by naturality of of $\mun$.

    For a polynomial $p$, define the multiplication at $p$ to be the image of the counit $\mcoun_{\free_p} \colon \free_{\free_p} \to \free_p$ under the isomorphisms
    \[
        \poly(\free_{\free_p}, \free_p) \iso \poly(\yon, [\free_{\free_p}, \free_p]) \iso \smcat^\sharp(\yon, \cofree_{[\free_{\free_p}, \free_p]}).
    \] 
    To show that the multiplication is natural we must show that for all polynomial $p$ and $q$, the following diagram commutes. 

    % https://q.uiver.app/#q=WzAsNyxbMSwwLCJcXGNvZnJlZV97W1xcZnJlZV9wLFxcZnJlZV9xXX0iXSxbMiwwLCJcXGNvZnJlZV97W1xcZnJlZV97XFxmcmVlX3B9LCBcXGZyZWVfe1xcZnJlZV9xfV19Il0sWzMsMCwiXFxjb2ZyZWVfe1tcXGZyZWVfe1xcZnJlZV9wfSwgXFxmcmVlX3tcXGZyZWVfcX1dfSBcXG90aW1lcyBcXGNvZnJlZV97W1xcZnJlZV97XFxmcmVlX3F9LCBcXGZyZWVfcV19Il0sWzIsMSwiXFxjb2ZyZWVfe1tcXGZyZWVfe1xcZnJlZV9wfSwgXFxmcmVlX3BdfVxcb3RpbWVzIFxcY29mcmVlX3tbXFxmcmVlX3AsIFxcZnJlZV9xXX0iXSxbMywxLCJcXGNvZnJlZV97W1xcZnJlZV97XFxmcmVlX3B9LCBcXGZyZWVfcV19Il0sWzAsMCwiXFxjb2ZyZWVfe1twLHFdfSJdLFswLDEsIlxcY29mcmVlX3tbXFxmcmVlX3AsIFxcZnJlZV9xXX0iXSxbMCwxXSxbMSwyXSxbMiw0XSxbMyw0XSxbNiwzXSxbNSw2XSxbNSwwXV0=
      \[\begin{tikzcd}
    {\cofree_{[p,q]}} & {\cofree_{[\free_p,\free_q]}} & {\cofree_{[\free_{\free_p}, \free_{\free_q}]}} & {\cofree_{[\free_{\free_p}, \free_{\free_q}]} \otimes \cofree_{[\free_{\free_q}, \free_q]}} \\
    {\cofree_{[\free_p, \free_q]}} && {\cofree_{[\free_{\free_p}, \free_p]}\otimes \cofree_{[\free_p, \free_q]}} & {\cofree_{[\free_{\free_p}, \free_q]}}
    \arrow[from=1-1, to=1-2]
    \arrow[from=1-1, to=2-1]
    \arrow[from=1-2, to=1-3]
    \arrow[from=1-3, to=1-4]
    \arrow[from=1-4, to=2-4]
    \arrow[from=2-1, to=2-3]
    \arrow[from=2-3, to=2-4]
  \end{tikzcd}\]

  It suffices to show that the following diagram commutes.
  % https://q.uiver.app/#q=WzAsNixbMCwwLCJcXGZyZWVfe1xcZnJlZV9wfSBcXG90aW1lcyBcXGNvZnJlZV97W1xcZnJlZV9wLCBcXGZyZWVfcV19Il0sWzAsMSwiXFxmcmVlX3AgXFxvdGltZXMgXFxjb2ZyZWVfe1tcXGZyZWVfcCwgXFxmcmVlX3FdfSJdLFsxLDEsIlxcZnJlZV9wIFxcb3RpbWVzIFtcXGZyZWVfcCwgXFxmcmVlX3FdIl0sWzEsMCwiXFxmcmVlX3tcXGZyZWVfcCBcXG90aW1lcyBbXFxmcmVlX3AsIFxcZnJlZV9xXX0iXSxbMiwxLCJcXGZyZWVfcSJdLFsyLDAsIlxcZnJlZV97XFxmcmVlX3F9Il0sWzAsMSwiXFxtY291bl97XFxmcmVlX3B9IFxcb3RpbWVzIFxcY29mcmVlX3tbXFxmcmVlX3AsIFxcZnJlZV9xXX0iLDJdLFsxLDIsIlxcZnJlZV9wIFxcb3RpbWVzIFxcY2NvdW5fe1tcXGZyZWVfcCwgXFxmcmVlX3FdfSIsMl0sWzAsMywiXFxtb2RzdHJ1Y3Rfe1xcZnJlZV9wLCBbXFxmcmVlX3AsIFxcZnJlZV9xXX0iXSxbMiwzLCJcXG11bl97XFxmcmVlX3AgXFxvdGltZXMgW1xcZnJlZV9wLCBcXGZyZWVfcV19Il0sWzMsNSwiXFxmcmVlX3tcXGV2YWx9Il0sWzUsNCwiXFxtY291bl97XFxmcmVlX3F9Il0sWzIsNCwiXFxldmFsIiwyXV0=
\[\begin{tikzcd}[column sep=huge,row sep=large]
	{\free_{\free_p} \otimes \cofree_{[\free_p, \free_q]}} & {\free_{\free_p \otimes [\free_p, \free_q]}} & {\free_{\free_q}} \\
	{\free_p \otimes \cofree_{[\free_p, \free_q]}} & {\free_p \otimes [\free_p, \free_q]} & {\free_q}
	\arrow["{\modstruct_{\free_p, [\free_p, \free_q]}}", from=1-1, to=1-2]
	\arrow["{\mcoun_{\free_p} \otimes \cofree_{[\free_p, \free_q]}}"', from=1-1, to=2-1]
	\arrow["{\free_{\eval}}", from=1-2, to=1-3]
	\arrow["{\mcoun_{\free_q}}", from=1-3, to=2-3]
	\arrow["{\free_p \otimes \ccoun_{[\free_p, \free_q]}}"', from=2-1, to=2-2]
	\arrow["{\mun_{\free_p \otimes [\free_p, \free_q]}}", from=2-2, to=1-2]
	\arrow["\eval"', from=2-2, to=2-3]
\end{tikzcd}\]

    The left-hand square commutes by definition of the interaction law $\modstruct$ and the right-hand square commutes by the zig-zag law of the adjunction.

    These maps satisfy the identity law because of the zig-zag laws of the adjunction. Multiplication satisfies associativity by naturality of the counit.
\end{proof}

\begin{remark}\label{rmk:free-lifts}
    The composite $U \circ \free \colon \poly \to \poly$ is a monad. 
    How is it related to our monad $\free \colon \org^\sharp \to \org^\sharp$? 
    Consider the map of enriching functors $\smcat^\sharp(\yon , -) \colon \smcat^\sharp \to \smset$. Under $\smcat^\sharp(\yon , -)$, the morphisms $\cofree_{[p,q]}$ in $\org^\sharp$ map to 
    \[
        \smcat^\sharp(\yon, \cofree_{[p,q]}) \iso \poly(\yon, [p,q]) \iso \poly(p, q).
    \]
    Therefore, under this map of enriching functors $\org^\sharp$ becomes $\poly$ and the monad $\free \colon \org^\sharp \to \org^\sharp$ becomes the monad $U \circ \free \colon \poly \to \poly$.
\end{remark}


\section{The operad $\org^\sharp_\free$}\label{sec:operad-orgm}

We begin by showing that $\org^\sharp$ has a symmetric monoidal structure $(\vee, 0)$ and that $\free \colon \org^\sharp \to \org^\sharp$ is lax monoidal with respect to $\vee$.

There is a symmetric monoidal structure $\vee$ on $\poly$ whose unit is $0$ and which is defined by 
\[
    p \vee q \coloneqq p + (p \otimes q) + q.
\] 



\begin{lemma}
    The functor $U \circ \free \colon \poly \to \poly$ is lax monoidal with respect to $\vee$.
\end{lemma}
\begin{proof}
    First we show by transfinite induction that there is a map $\free_p \to [ \free_{q}, \free_{p \vee q}]$. It suffices to show that for each ordinal $\alpha$ there is a map $p\hoc{\alpha} \otimes \free_q \to \free_{p \vee q}$ such that for all ordinals $\alpha$, 
    % https://q.uiver.app/#q=WzAsMyxbMCwwLCJwXFxob2N7XFxhbHBoYX1cXG90aW1lcyBcXGZyZWVfcSJdLFswLDEsInBcXGhvY3tcXGFscGhhKzF9IFxcb3RpbWVzIFxcZnJlZV9xIl0sWzEsMSwiXFxmcmVlX3twIFxcdmVlIHF9Il0sWzAsMSwiXFxpb3RhIFxcaG9je1xcYWxwaGF9IiwyXSxbMCwyXSxbMSwyXV0=
\begin{equation}\label{eq:p_freeq_freepveeq}
\begin{tikzcd}
	{p\hoc{\alpha}\otimes \free_q} \\
	{p\hoc{\alpha+1} \otimes \free_q} & {\free_{p \vee q}}
	\arrow["{\iota \hoc{\alpha}}"', from=1-1, to=2-1]
	\arrow[from=1-1, to=2-2]
	\arrow[from=2-1, to=2-2]
\end{tikzcd}    
\end{equation}

    \begin{itemize}
        \item Base case. We define the map $p\hoc{0} \otimes \free_q = \yon \otimes \free_q \to  \free_{p \vee q}$ to be induced by $\free$ applied to the inclusion $q \to p \vee q$.

        \item For successor ordinals $\alpha +1$,
        \begin{align*}
            p\hoc{\alpha + 1} \otimes \free_q  & = (\yon + p \tri p\hoc{\alpha})\otimes \free_ q \\
            & \to \free_q  + p \tri(p \hoc{\alpha} \otimes \free_q \\
            & \to \free_q + p \tri \free_{p \vee q}\\
            & \to \free_{p \vee q} + (p \vee q) \tri \free_{p \vee q}\\
            & \iso \free_{p \vee q}.
        \end{align*}
        follows by duoidality and the induction hypothesis. 

        That these maps are natural with respect to the inclusions $\iota\hoc{\alpha} \colon p\hoc{\alpha} \to p \hoc{\alpha + 1}$ is straightforward by induction.
        
        \item For limit ordinals $\alpha$, define 
        $\left(\colim_{\alpha' < \alpha} p\hoc{\alpha'}\right) \otimes \free_q \to \free_{p \vee q}$  to be the universal map induced by the components $p\hoc{\alpha'} \otimes \free_q \to \free_{p \vee q}$ given by the induction hypothesis. 

        Lastly we must show that the diagram in \cref{eq:p_freeq_freepveeq} commutes. By definition of $\iota\hoc{\alpha}$ it suffices to show that for all $\alpha' < \alpha$, the following diagram commutes. 

        % https://q.uiver.app/#q=WzAsNSxbMCwwLCIoXFx5b24gKyBwIFxcdHJpIHAgXFxob2N7XFxhbHBoYSd9KSBcXG90aW1lcyBcXGZyZWVfcSJdLFswLDEsIlxcbGVmdChcXHlvbiArIHAgXFx0cmkgIFxcbGVmdChcXGNvbGltX3tcXGFscGhhJyA8IFxcYWxwaGF9IHAgXFxob2N7XFxhbHBoYSd9IFxccmlnaHQpXFxyaWdodCkgXFxvdGltZXMgXFxmcmVlX3EiXSxbMSwxLCJcXGZyZWVfcSAgKyBwIFxcdHJpIFxcY29saW1fe1xcYWxwaGEnIDwgXFxhbHBoYX0gcCBcXGhvY3tcXGFscGhhJ30gXFxvdGltZXMgXFxmcmVlX3EiXSxbMSwwLCJcXGZyZWVfcSArIHAgXFx0cmkgKHAgXFxob2N7XFxhbHBoYSd9IFxcb3RpbWVzIFxcZnJlZV9xfSJdLFsyLDEsIlxcZnJlZV97cCBcXHZlZSBxfSJdLFswLDFdLFswLDNdLFsxLDJdLFszLDJdLFsyLDRdLFszLDRdXQ==
\[\begin{tikzcd}
	{(\yon + p \tri p \hoc{\alpha'}) \otimes \free_q} & {\free_q + p \tri (p \hoc{\alpha'} \otimes \free_q} \\
	{\left(\yon + p \tri  \left(\colim_{\alpha' < \alpha} p \hoc{\alpha'} \right)\right) \otimes \free_q} & {\free_q  + p \tri \colim_{\alpha' < \alpha} p \hoc{\alpha'} \otimes \free_q} & {\free_{p \vee q}}
	\arrow[from=1-1, to=1-2]
	\arrow[from=1-1, to=2-1]
	\arrow[from=1-2, to=2-2]
	\arrow[from=1-2, to=2-3]
	\arrow[from=2-1, to=2-2]
	\arrow[from=2-2, to=2-3]
\end{tikzcd}\]
        The left-hand square commutes by naturality of duoidality and the distributive law. The right-hand triangle commutes by definition of $\left(\colim_{\alpha' < \alpha} p \hoc{\alpha'}\right) \otimes \free_q \to \free_{p \vee q}$.
    \end{itemize}

    The induced map $\free_p \otimes \free_q \to \free_{p \vee q}$ along with the maps $\free_p \to \free_{p \vee q}$ and $\free_q \to \free_{p \vee q}$ given by $\free$ applied to the inclusions $p \to p \vee q$ and $q \to p \vee q$, togther define the laxator 
    \[
        \free_p \vee \free_q \to \free_{p \vee q}.
    \]
    The unitor is the unique map $\mun_0 = ! \colon  0 \to \free_0$.

    Associativity is straightforward by transfinite induction and  unitality is immediate. 
\end{proof}


We can extend the symmetric monoidal structure $\vee$ to $\org^\sharp$ as follows.
The monoidal structure on objects is the same as that for $\poly$.
We define the monoidal structure on morphisms in three components. First, there is a map 
\[
    p \otimes \cofree_{[p,q]} \otimes \cofree_{[p', q']} \to p \otimes \cofree_{[p,q]} \xrightarrow{p \otimes \ccoun_{[p,q]}} p \otimes [p,q] \to q \to q \vee q'.
\] where the first map in the composite is induced by the counit of the $\tri$-comonoid $\cofree_{[p', q']}$. Likewise there is a map $p' \otimes \cofree_{[p,q]} \otimes \cofree_{[p', q']} \to q \vee q'$. Lastly, there is a map 
\[
    p \otimes p' \otimes \cofree_{[p,q]} \otimes \cofree_{[p', q']} \xrightarrow{p \otimes p' \otimes \ccoun_{[p,q]} \otimes \ccoun_{[p', q']}} p \otimes p' \otimes [p,q] \otimes [p', q'] \to q \otimes q' \to q \vee q'.
\] Together these define a unique map 
\[
    (p \vee p')\otimes \cofree_{[p,q]} \otimes \cofree_{[p', q']} \to q \vee q'.
\]
The action of the monoidal product on morphisms is the image of this map under the isomophisms,
\begin{align*}
    \poly((p \vee p')\otimes \cofree_{[p,q]}\otimes \cofree_{[p', q']}, q \vee q') &\iso \poly(\cofree_{[p,q]}\otimes \cofree_{[p', q']}, [p \vee p', q \vee q']) \\
    &\iso \smcat^\sharp(\cofree_{[p,q]}\otimes \cofree_{[p', q']}, \cofree_{ [p \vee p', q \vee q']}).
\end{align*}

The associator, left and right unitors, triangle identity, and pentagon identity all follow directly from the fact that $\vee$ with unit $0$ is a symmetric monoidal structure on $\poly$ \cite{spivak2022reference}.


    

\begin{theorem}
    The $\smcat^\sharp$-enriched functor $\free \colon \org^\sharp \to \org^\sharp$ is lax monoidal with respect to the monoidal structure $\vee$.
\end{theorem}
\begin{proof}
    The unitor is the image of the unitor of $\free \colon \poly \to \poly$ under the isomorphisms
    \[
        \poly(0, \free_0) \iso \poly(\yon , [0, \free_0]) \iso \smcat^\sharp(\yon, \cofree_{[0, \free_0]}).
    \]
    The laxator is the image of the laxator of $\free \colon \poly \to \poly$ under the isomorphisms
    \[
        \poly(\free_p \vee \free_q, \free_{p \vee q}) \iso \poly(\yon, [\free_p \vee \free_q, \free_{p \vee q}]) \iso \smcat^\sharp(\yon, \cofree_{[\free_p \vee \free_q, \free_{p \vee q}]}).
    \]
    Associativity and unitality follow directly.
\end{proof}

\begin{definition}
    Let $\org^\sharp_\free$ be the $\smcat^\sharp$-enriched operad underlying the Kleisli category $\org^\sharp_\free$. The objects of $\org^\sharp_\free$ are polynomials and its morphisms are defined by 
    \[
        \org^\sharp_\free(p_1 , \cdots, p_m; q) = \cofree_{[p_1 \vee \cdots \vee p_m, \free_q]}.
    \]

    Let $(\org^\sharp_\free)\op$ be the $\smcat^\sharp$-enriched operad underlying the opposite category $(\org^\sharp_\free)\op$. The objects of $(\org^\sharp_\free)\op$ are polynomials and its morphisms are defined by 
    \[
        (\org^\sharp_\free)\op(q_1, \cdots , q_m; p) = \cofree_{[p, \free_{q_1 \vee \cdots \vee q_m}]}.
    \]

    Let $\org_\free$ and $\org_\free\op$ be the corresponding $\smcat$-enriched categories under the map of enriching categories $\Cat{CAT}(-,\smset) \colon \smcat^\sharp \to \smcat$.
\end{definition}




The $\smcat$-enriched category $\org_\free\op$ has polynomials as objects and a morphism in $\org_\free\op(p_1, \cdots, p_m; q)$ is a $[q, \free_{p_1 \vee \cdots \vee p_m}]$-coalgebra. We can think of such a coalgebra as a dynamically changing strategy that answers $q$-shaped questions using $p_i$-shaped subordinates. Given a position of $q$ (which we think of as a $q$ question) the state of the coalgebra determines a decision tree made up of $p_i$ components. At each fork of the decision tree one or more $p_i$ questions may be asked. Each possible either either returns an answer to the $q$ question and updates the state or determines another set of $p_i$ questions to be asked. 

In the $\smcat$-enriched operad $\org$ defined in \cite{XXX}, each subordinate must be consulted exactly once and the answers from each subordinate aggregated into a single answer. The morphisms of $\org_\free\op$ are more general since subordinates may be  consulted one or many times strategically as a result of previous subordinate answers. 

\begin{example}\label{ex:stream-m}



Consider the polynomials \textcolor{cyan}{$\alice = \yon^2$}, \textcolor{magenta}{$\bob = \yon^2$}, and \textcolor{teal}{$\carmen = \yon^2$}. These polynomials represent subordinates Alice, Bob, and Carmen who when consulted return one of two choices. 
A coalgebra in $\org_\free\op(\alice, \bob, \carmen; \yon^2)$ will consult Alice, Bob, and Carmen in order to produce one of two choices. 

To begin, we might consider the coalgebra with a single state that is defined by the polynomial map $\yon^2 \to \free_{\alice \vee \bob \vee \carmen}$ that maps the single position of $\yon^2$ to the following decision tree
\[
\begin{tikzpicture}[trees,
   level distance=1.5cm,sibling distance=1.75cm, 
   edge from parent path={(\tikzparentnode) -- (\tikzchildnode)}]
\Tree [.$\bullet$
    \edge node[auto=left] {$(\azero, \bzero)$};  
    [.$\quad$ ]
    \edge node[auto=right] {$(\azero, \bone)$};
    [.$\bullet$
      \edge node[auto=left] {$\czero$};
      [.$\quad$ ] 
      \edge node[auto=right] {$\cone$};
      [.$\quad$ ] 
    ] 
    \edge node[auto=left] {$(\aone, \bzero)$};
    [.$\bullet$
      \edge node[auto=left] {$\czero$};
      [.$\quad$ ] 
      \edge node[auto=right] {$\cone$};
      [.$\quad$ ] 
    ] 
    \edge node[auto=right] {$(\aone, \bone)$};  
    [.$\quad$ ]
    ]
\end{tikzpicture}
\]
and maps the directions from left to right to $0$, $0$, $1$, $0$, $1$, and $1$. In other words, Alice and Bob are both asked for a $0$ or $1$. If their responses agree then that value is returned. Otherwise, Carmen is the tie-breaker. 

We can upgrade this coalgebra to have non-trivial dynamics. Consider a coalgebra with three states. The first states takes the single position of $\yon^2$ to the decision tree as above. The second state takes the single position of $\yon^2$ to a similar decision tree except where Alice is the tie-breaker instead of Carmen. The third state takes the single position of $\yon^2$ to the decision tree where Bob is the tie-breaker. On directions, if Carmen tie-breaks then the state updates so that Alice is the new tie-breaker. Once Alice tie-breaks, then the state updates so that Bob is the new tie-breaker. And so on.

A more complicated dynamics might learn a preference over initial consultants versus tie-breakers based on the quality of the results. 

If the subordinates Alice, Bob, and Carmen also dynamically decide between $0$ and $1$ based on a decision tree of their own subordinate answers, then composition in $\org_\free\op$ defines a coalgebra which operates on these sub-subordinates. 
\end{example}