\chapter{Background} \label{sec:background}

For background on $\poly$, including the monoidal closed structure $(\otimes,[-,-])$, see \cite{niu2022poly}. In \cref{sec.(co)free(co)monad} we briefly recall the free monad $\free_-$ and cofree comonad $\cofree_-$ constructions, as well as the module structure $\free_p\otimes\cofree_q\to\free_{p\otimes q}$ between them. In \cref{sec.orgsharp} we discuss the $\smcat^\sharp$-enriched category $\org^\sharp$ to which we will extend the free monad and cofree comonad constructions.


\section{The free monad monad and the cofree comonad comonad}\label{sec.(co)free(co)monad}

Two key players in our story are the free monad monad $\free$ and the cofree comonad comonad $\cofree$. The general construction of the free monad monad was introduced in \cite{kelly1980unified} (see also \cite{nlab:transfinite_construction_of_free_algebras}) and an intuitive explanation was given in \cite{libkind2024pattern}; we describe their key features below. A \defined{polynomial monad} (referred to here as monad) is a $\tri$-monoid. The category of polynomial monads is denoted $\modpoly$. A \defined{ polynomial comonad} (referred to here as comonad) is a $\tri$-comonoid. There is an equivalence
\[
    \Cat{Comod}_\poly\cong\smcat^\sharp
\]
between the category of polynomial comonads and the category $\smcat^\sharp$ consisting of categories and retrofunctors (sometimes called cofunctors; see \cite{aguiar1997internal,clarke2022introduction,nlab:retrofunctor}) between them.

Given a polynomial $p$, the associated free monad is the polynomial $\free_p$ whose positions are $p$-shaped decision trees and whose directions are the leaves of the tree. For example, the following is a position of $\free_{\yon^2}$ with $5$ directions. 

\[
\begin{tikzpicture}[trees, thick, level distance=0.5cm,
  level 1/.style={sibling distance=10mm}]


    \node (b) {}
        child {
            child {node (b1) {}}
                child {
                    child {
                        child {node (b2) {}}
                        child {node (b3) {}}
                    }
                    child {node (b4) {}}
                }
        }
        child {node (b5) {}}
    ;

\end{tikzpicture}
\]

The free monad is defined in \cite{libkind2024pattern} via transfinite induction where we define polynomials $p\hoc{\alpha}$ for ordinals $\alpha$ and cartesian inclusions $\iota\hoc\alpha\colon p\hoc{\alpha} \to p\hoc{\alpha +1}$. It is given in the base case by $p\hoc{0} \coloneqq \yon$, for successor ordinals by $p\hoc{\alpha + 1} \coloneqq \yon + p \tri p\hoc{\alpha}$, and for limit ordinals $\alpha$ by $p\hoc{\alpha} \coloneqq \colim_{\alpha' < \alpha} p \hoc{\alpha'}$. 


\cite[Theorem 2.10]{libkind2024pattern} defines an adjunction
\[
    \adj{\poly}{\free_-}{U}{\modpoly}
\]
whose unit we denote $\mun$ and whose counit we denote $\mcoun$.

\begin{example}
The free monad on a set $A$ is the $A$-exceptions monad
\[\free_A=\yon+A.\]
In particular, $\free_0=\yon$. The free monad on $A\yon$ is $\free_{A\yon}=\List(A)\yon$, e.g.\ $\free_\yon=\nn\yon$.
\end{example}

Given a polynomial $p$, the associated cofree comonad is $\cofree_p$ whose positions are $p$-shaped behavior trees and whose directions are finite paths up the tree. 
\cite[Theorem 3.2]{libkind2024pattern} defines an adjunction 
\begin{equation}\label{eqn.cofree_adjunction}
    \adj{\catsharp}{U}{\cofree_-}{\poly}
\end{equation}
whose counit we denote $\ccoun$. The functor $\cofree$ is lax monoidal with respect to $\otimes$, so it has a productor and unitor of the following types:
\begin{equation}\label{eqn.cofree_lax_monoidal}
\cofree_p\otimes\cofree_q\to\cofree_{p\otimes q}\qqand\yon\to\cofree_\yon.
\end{equation}

\begin{example}
The cofree comonad on a set $A$ is $\cofree_A=A\yon$. The cofree comonad on a representable $\yon^A$ is $\cofree_{\yon^A}=\yon^{\List(A)}$. Finally, the cofree comonad on $A\yon$ is $\cofree_{A\yon}=(A\yon)^\nn$. It has as positions all streams in $A$, and a direction is a natural number: the direction $n$ corresponds to the first $n$ tokens of the stream.

\end{example}

The main result of \cite[Theorem 3.4]{libkind2024pattern} was to prove that $\free_-$ is a left module over $U \circ  \cofree \colon \poly \to \poly$ whose action is defined by the natural transformation 
\[
    \modstruct_{p,q}\colon \free_p \otimes \cofree_q \to \free_{p \otimes q}
\] which is defined as the image of the composite
\[
    p \otimes \cofree_q \xrightarrow{p \otimes \ccoun_q} p \otimes q \xrightarrow{\mun_{p \otimes q}} \free_{p \otimes q}
\] under the isomorphisms
\[
    \poly(p \otimes \cofree_q, \free_{p\otimes q})  \iso \modpoly(\free_p, [\cofree_q, \free_{p\otimes q}]) \to \poly(\free_p \otimes \cofree_q, \free_{p \otimes q}).
\] We call $\modstruct$ the \defined{interaction law}.



\section{The $\smcat^\sharp$-enriched category $\org^\sharp$}\label{sec.orgsharp}

In this paper we define $\org^\sharp$ to be the $\smcat^\sharp$-enriched category whose objects are polynomials and whose morphisms are defined by $\org^\sharp(p, q) \coloneqq \cofree_{[p,q]}$, the cofree comonad on the internal hom $[p,q]$. 
Composition is defined by 
\[
    \cofree_{[p,q]} \otimes \cofree_{[q,r]} \to \cofree_{[p, q] \otimes [q, r]} \to \cofree_{[p, r]}
\]
where the first map is the productor \eqref{eqn.cofree_lax_monoidal} of $\cofree$ and the second uses functoriality of $\cofree$ and internal-hom composition.

For the identity on a polynomial $p$, we use the fact that $\yon$ has a unique comonoid structure, and take the image of the polynomial identity on $p$ under the isomorphisms
\[
    \poly(p, p) \iso \poly(\yon , [p, p]) \iso \smcat^\#(\yon, \cofree_{[p,p]}).
\]

Recall that for any orthogonal factorization system $(\cat{L},\cat{R})$ on a category $\cat{C}$, there is a functor $(\cat{R}/-)\colon\cat{C}\to\smcat$ sending $c$ to the slice category $\cat{R}/c$ of maps $d\To{\rho}c$ and sending $f\colon c\to c'$ to the functor $\rho\mapsto\rho'$ where $\rho'$ is the right-factor of the composite $\rho\then f=\ell\then\rho'$. If $\cat{C}$ has a monoidal structure $\otimes$ that lifts to a monoidal structure on $\cat{L}$ and $\cat{R}$, then $(\cat{R}/-)$ is lax monoidal
\[
\big((d\To{\rho}c), (d'\To{\rho'}c')\big)\mapsto (d\otimes d'\To{\rho\otimes\rho'}c\otimes c').\]
Recall also \cite[Proposition 5.52]{niu2022poly} that there is a (vertical, cartesian) orthogonal factorization system on $\smcat^\sharp$ and that cartesian retrofunctors over $\cat{C}$ can be identified with discrete opfibrations over $\cat{C}$, or equivalently to functors $\cat{C}\to\smset$. By the previous paragraph, we have a lax monoidal functor
\[
%(\smcat^{\sharp\tn{cart}}/-)
\smset^-\colon(\smcat^\sharp,\yon,\otimes)\too(\smcat,1,\times)
\]
This in turn provides a change of enrichment, which sends $\org^\sharp$ to the monoidal 2-category whose objects are polynomials under $\otimes$ and for which $\hom(p,q)=\smset^{\cofree_{[p,q]}}$. This is equivalent to the category-enriched operad $\org$ defined in \cite[Definition 2.19]{spivak2021learnersv1}.

\begin{remark}
    What is the relationship between $\poly$ and $\org^\sharp$? Since $(\poly, \otimes, \yon)$ has internal homs, $\poly$ can be viewed as a $(\poly, \otimes, \yon)$-enriched category. Via the lax monoidal functor of enrichment categories $\cofree\colon (\poly, \otimes, \yon) \to (\smcat^\sharp, \otimes, \yon)$, we obtain the $\smcat^\sharp$-enriched category $\org^\sharp$.
\end{remark}

We will not need the following proposition, but we record it here both as an interesting note and to settle the various notions for our readers.
\begin{proposition}
Both $(\org^\sharp,\yon,\otimes)$ and $(\org,\yon,\otimes)$ are monoidal closed.
\end{proposition}
\begin{proof}
For any objects $p,q:\ob(\poly)=\ob(\org^\sharp)=\ob(\org)$, the associated hom-objects for $\org^\sharp$ and $\org$ are
\[
  \org^\sharp(p,q)\coloneqq\cofree_{[p,q]}
  \qqand
	\org(p,q)\coloneqq\smset^{\cofree_{[p,q]}}=[p,q]\coalg.
\]
In either case, the monoidal closure is given by the internal hom $[-,-]$ because we have natural isomorphisms (in $\catsharp$ and $\Cat{CAT}$ respectively) of the form:
\[
	\cofree_{[p\otimes p',q]}\cong\cofree_{[p,[p',q]]}
	\qqand
	[p\otimes p',q]\coalg\cong[p,[p',q]]\coalg.
	\qedhere
\]
\end{proof}

